\newgeometry{left=18mm,right=18mm, top=16mm, bottom=16mm}
\thispagestyle{empty}
\section*{Abstract}
The nucleosynthesis of heavy elements (A > 56) beyond iron, primarily via the rapid neutron-capture process (r-process), represents one of the most fascinating and complex phenomena in nuclear astrophysics. This process, which is responsible for the majority of heavy element production in the universe, provides a critical interface between astrophysics and nuclear physics, offering a unique "laboratory" to study nuclear reactions and properties far from stability. Neutron stars (NS), with their extreme densities and exotic compositions, are of exceptional interest in this context. These astrophysical objects serve as a natural laboratory to explore nuclear structure under extreme conditions, such as those described by the equation of state (EoS). Understanding the EoS, which governs the macroscopic properties of neutron stars, remains a formidable challenge. While astrophysical observations provide macroscopic constraints on NS models, nuclear experiments in terrestrial laboratories play a vital role in constraining the microscopic nuclear physics parameters embedded within these models.
Moreover, neutron star mergers (NSMs), which are stellar collisions involving NSs, are recognized as the primary astrophysical sites for the r-process. These events generate conditions favorable for the production of neutron-rich nuclei, driving the synthesis of heavy elements. A comprehensive understanding of the r-process requires detailed knowledge of nuclear structure and reaction dynamics in regimes far from nuclear stability. However, experimental constraints on critical observables, such as fission yields and fission barriers, are limited due to the challenges associated with producing and studying these neutron-rich nuclei in laboratory settings.
This doctoral thesis, titled "Nuclear Structure Investigations of Light Nuclei with the R3B Experiment", addresses these challenges by presenting the analysis of two key reaction channels using carbon isotopes. These investigations were conducted with the R3B (Reactions with Relativistic Radioactive Beams) setup during the commissioning experiment S444 as part of the FAIR Phase-0 campaign at GSI. Specifically, this work focuses on:
\begin{enumerate}
\item Charge-changing and total interaction cross sections: The study of 12C+12C collisions via the transmission method. These measurements provide critical insights into the nuclear structure and reaction mechanisms of light nuclei, which serve as essential benchmarks for understanding nuclear matter in astrophysical environments.
\item Quasi-free scattering (QFS) reactions: The investigation of the 12C(p,2p)11B reaction as a tool to probe nuclear structure. This approach demonstrates its potential for studying the dynamics of the r-process through fission in QFS experiments, offering an innovative method to address experimental limitations in current r-process studies.
\end{enumerate}
The results of these analyses contribute to our understanding of the structure of light nuclei and provide critical experimental constraints relevant to nuclear astrophysics. By advancing our knowledge of fundamental nuclear properties and reaction dynamics, this work bridges the gap between laboratory-based nuclear physics and astrophysical processes, shedding light on the origin of heavy elements in the universe.
