\newgeometry{left=18mm,right=18mm, top=16mm, bottom=16mm}
\thispagestyle{empty}
\section*{Abstract}
Astrophysical observations of neutron stars (NS) provide rather precise data about the global properties of such unique and fascinating objects, e.g., the mass and radius. For the interpretation of this data and to gain a profound understanding of the inner structure of NS, it is essential to investigate nuclear matter under extreme conditions. A key instrument to describe nuclear matter over a wide density range is the equation of state (EOS). The radius and stability of NS are governed by the pressure of the highly asymmetric matter in the inside, which is defined by the so-called symmetry energy in the EOS. This quantity and especially its slope parameter L around nuclear saturation density are experimentally only weakly constraint so far.\newline
A experimental attempt to constrain the L parameter is to measure the neutron-skin thickness of highly asymmetric nuclei since both quantities are directly correlated. One of the most established experimental methods to probe the nuclear density distribution is the measurement of total interaction cross section at radioactive beam facilities. A common method to describe integrated cross sections is the Glauber reaction model. In such a model, which includes realistic in-medium modification for composite nuclei, the only inputs are the experimental nucleon-nucleon cross sections and the density distribution of the projectile and target nucleus.\newline
For a precise determination of the neutron-skin thickness of exotic nuclei, and thus to constrain the symmetry energy slope parameter, it is essential to quantify the uncertainty of the reaction model under ideal conditions. \newline
This work provides a detailed summary of the precise measurement of total interaction cross sections of \ce{^{12}C}+\ce{^{12}C} collisions in the energy regime between 400 and 1000 MeV/nucleon. The underlying experiment was carried out as part of the commissioning of the R\textsuperscript{3}B setup during the FAIR Phase-0 campaign at GSI. The present analysis of total interaction cross sections is based on a transmission measurement, where the numbers of incoming and non-reacted projectiles before and after the reaction target have been identified. The identification of the non-reacted \ce{^{12}C} poses a challenge to the experimental setup since the time- and rate-dependent detector efficiency, as well as the geometrical acceptance of the whole setup, have to be considered.\newline
The presented cross sections was determined with a total experimental uncertainty down to 0.4 \% and represent the most precise data currently available in this energy regime. The validity of the measurement and analysis method was confirmed by data from previous experiments. It was shown that predictions based on a realistic Glauber reaction model are in good agreement with the presented experimental results for low energy but overestimate them by around 2.5 \% at higher energies. 