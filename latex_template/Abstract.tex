\newgeometry{left=18mm,right=18mm, top=16mm, bottom=16mm}
\thispagestyle{empty}
\section*{Abstract}
The nucleosynthesis of heavy elements (A > 56) beyond iron, primarily via the rapid neutron-capture process (r-process), represents one of the most fascinating and complex phenomena in nuclear astrophysics. This process, which is responsible for the majority of heavy element production in the universe, provides a critical interface between astrophysics and nuclear physics, offering a unique "laboratory" to study nuclear reactions and properties far from stability. Moreover, neutron star mergers (NSMs), which are stellar collisions involving NSs, are recognized as major astrophysical sites for the r-process. Neutron stars (NS), with their extreme densities and exotic compositions, are of exceptional interest in this context. These astrophysical objects serve as a natural laboratory to explore nuclear structure under extreme conditions, such as those described by the equation of state (EoS). Understanding the EoS, which governs the macroscopic properties of neutron stars, remains a formidable challenge. While astrophysical observations provide macroscopic constraints on NS models, nuclear experiments in terrestrial laboratories play a vital role in constraining the microscopic nuclear physics parameters embedded within these models. A comprehensive understanding of the r-process requires detailed knowledge of nuclear structure and reaction dynamics in regimes far from nuclear stability.\newline
%However, experimental constraints on critical observables, such as fission yields and fission barriers, are limited due to the challenges associated with producing and studying these neutron-rich nuclei in laboratory settings.
This doctoral thesis, titled "Nuclear Structure Investigations of Light Nuclei with the R$^3$B Experiment", addresses detailed investigations on the state of the art experimental techniques to study key properties of the most exotic nuclei with highest precision. These investigations were conducted with the R$^3$B (Reactions with Relativistic Radioactive Beams) setup during the commissioning experiment S444 as part of the FAIR Phase-0 campaign at GSI. Specifically, this work focuses on:
\begin{enumerate}
\item Charge-changing and total interaction cross sections: The study of $^{12}$C+$^{12}$C collisions via the transmission method. These measurements provide critical insights into the nuclear matter radius and its distribution, which serve as essential benchmarks for understanding the EoS of nuclear matter in astrophysical environments.
\item Quasi-Free Scattering (QFS) reactions: The investigation of the $^{12}$C(p,2p)$^{11}$B reaction as a tool to probe the single particle structure of nuclei. This approach demonstrates in addition its potential for studying the recycling branch of the r-process through fission in QFS experiments and with it probing the evolution of fission barriers so far only known for few nuclei close to stability. 
\end{enumerate}
By advancing our knowledge of fundamental nuclear properties and reaction dynamics, this work bridges the gap between laboratory-based nuclear physics and astrophysical processes, shedding light on the origin of heavy elements in the universe.
