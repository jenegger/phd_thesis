\newgeometry{left=18mm,right=18mm, top=16mm, bottom=16mm}
\thispagestyle{empty}
\section*{Abstract}
-you can study to constrain the L parameter  in the EOS and therefore gain profound understanding of the inner structure of NS.\newline
At GSI the 12C reaction on carbon and plastic was studied. This allows to study the microstopic structure and therefore the composition of nuclear matter and the acting force (nuclear shell model etc) by using the method of QFS.\newline
This qfs method, which offers the possibility to look inside the nucleus can be used thanks to the newly designed calorimeter for the detection of gammas and light charged particles CALIFA for the R3B experiment at GSI.\newline
This work provides an overview of the precise total reaction cross section measurement of 12C +12C collisions for the experiment in inverse kinematics at beam energies from 400 AMeV up to 800 AMeV. As well as looking at the qfs method to extract valuable information of the inner structure of the 12C nucleus.\newline
-to add: what methods did I use\newline
-impact in the wider context\newline
-limitations and future directions\newline
