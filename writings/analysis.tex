\section{Analysis - Total Interaction Cross Section of  $^{12}$C + $^{12}$C}
This chapter will go through the  analysis step by step from the unpacking stage to the final measurement of the total interaction cross section. It will start by a short overview of the transmission method used for the cross section measurements. The next step is  the selection of clean incoming $^{12}$C isotopes. Following the identification of the carbon isotopes after the target - for the measurement of the charge changing cross section - and as final step the interaction cross section measurement. \newline
All relevant detector related geometrical and efficiency corrections will be adressed and their influence to the final result and its uncertainty will be discussed.
\subsection{Cross Section Measurement via Transmission Method}\label{section:transmission_method}
In its most generic form cross sections give a measure of the probability that a specific reaction will take place when two or more particles collide. The cross sections  measured in scattering exepriments, as well as the energy and angular distribution of the reaction products, provide information about the dynamics of the interaction between the projectile and the target particle, i.e., about the shape of the interaction potential and the coupling strength.\newline
The cross section $\sigma$ can be derived by looking at the relation between the number of incoming particles (N$_{1}$) and unreacted particles after the target ($N_{2}$). For an experiment with fixed target with thickness $z$ and volumetric number density $n$ the number of reacted particles in the infinitesimal thin target layer $dz$ can be expressed as:
\begin{equation}
\frac{dN_{2}}{dz} = -n \sigma N_{2}
\end{equation}
Solving this differential equation for $N_{2}$ (with the condition $N_{2}$ = $N_{1}$ for $z=0$) discloses an exponential relation:
\begin{equation}
N_{2} = N_{1}e^{-n\sigma z}
\label{eq:cross_sec}
\end{equation} 
Where $n\cdot z$ can be summarized as $N_t$, the total number of scattering centers per unit area. The relation $(N_{2}/N_{1})$, number of unreacted particles after the target versus number of incoming particles, is often called survival probability. For an idealistic experimental setup with full detector efficiency and no interactions in the setup material the cross section could simply be deduced from equation \ref{eq:cross_sec}. To account for reactions of the projectile that occur within the setup material and first order detector specific distortions of output signals the survival probabiltiy $(N_{2}/N_{1})$ has to be divided by $(N_{2}^E/N_{1}^E)$, where $N_{1}^E$ is the number of incoming particles and $N_{2}^E$ the number of unreacted particles after the target for an empty run respectively. Thereby the setup specific efficiency($\epsilon_{setup}$) and transmission factor($t_{setup}$) are cancelled out to obtain the underlying number of unreacted particles after the target $\tilde{N_{2}}$:\newline
$N_{2} = \tilde{N_{2}} \cdot t_{setup}\cdot \epsilon_{setup}$, with $\tilde{N_{2}}$\newline
$N_{2}^E = \tilde{N_{2}^E} \cdot \epsilon_{setup}$ with $\frac{\tilde{N_{2}^E}}{N_{1}^E}$ the setup specific transmission factor $t_{setup}$ \newline
\newline
The final formula for the cross section for a so called transmission measurement is:
\begin{equation}
	\begin{split}
\sigma = -\frac{1}{N_t} ln(\frac{N_{1}^E}{N_{2}^E} \cdot \frac{N_2}{N_1}) = -\frac{1}{N_t} ln(\frac{N_{1}^E}{\tilde{N_{2}^E} \cdot t_{setup}\cdot \epsilon_{setup}} \cdot \frac{\tilde{N_{2}^E} \cdot \epsilon_{setup}}{N_1}) \\
\text{With $\frac{\tilde{N_{2}^E}}{N_{1}^E} = t_{setup}$} \\
\sigma = -\frac{1}{N_t} ln(\frac{1}{t_{setup}} \cdot \frac{\tilde{N_{2}} \cdot t_{setup}}{N_1}) = -\frac{1}{N_t} ln(\frac{\tilde{N_{2}}}{N_1})
\label{eq:corr_cross}
	\end{split}
\end{equation}
From the above formula \ref{eq:corr_cross} it is evident that for cross section measurements with the transmission method three types of observables have to be measured:\newline
\begin{enumerate}
\item[$\blacksquare$] \textbf{Number of scattering centers $\mathbf{N_t}$}\newline
The number of scattering centers per unit area of the target is a target specific number. It depends from the target thickness and and its density. The values herefore are taken from \cite{ponnath2023precise}\footnote{For the purpose of this work the target thicknesses were remeasured at GSI with a chromatic sensor giving 2D depth profiles of each target.}:
\begin{enumerate}
\item Thin target:\newline 
target thickness d = 0.5451 cm; $N_t = 5.0588795\cdot 10^{22}$; $\Delta N_t = 0.0648\%$
\item Medium target:\newline
target thickness d = 1.0793 cm; $N_t = 1.0016600\cdot 10^{23}$; $\Delta N_t = 0.2620\%$
\item Thick target:\newline
target thickness d =2.1928cm; $N_t = 2.0350598\cdot 10^{23}$; $\Delta N_t = 0.0322\%$
\end{enumerate} 
where $N_t$ was calculated by:
\begin{equation}
N_t = \frac{\rho \cdot d \cdot N_A}{M}
\end{equation}
with $\rho$ the target density\footnote{$\rho = 1.851 g/cm^{3}$, from \cite{ponnath2023precise}}, $N_A$ the Avogadro constant ($6.02214076\cdot10^{23}\,mol^{-1}$) and $M$ the molar mass of the target (for carbon $M = 12.011g \cdot mol^{-1}$).
\item[$\blacksquare$] \textbf{Number of incoming projectiles ($^{12}$C) $\mathbf{N_1}$}\newline
For the measurement only events with well identified incoming $^{12}$C projectiles are chosen. Herefore strict cuts on the detectors upstream the target area are set. This strict event selection makes sure that we only consider events with single $^{12}$C. This will be discussed in more detail in section \ref{subsec:event-sel}.  %TODO finish this sencence 
\item[$\blacksquare$] \textbf{Number of unreacted projectiles ($^{12}$C) $\mathbf{N_2}$ after the target}\newline
Detectors downstream the target area are used to count the number of unreacted projectiles $^{12}$C. To reduce detector specific influences which could distort the result it is advisable to use only as few as requirable detectors for the clear identification of unreacted projectiles. Moreover detector specific efficiencies are cancelled out by including both empty and target runs in the cross section calculation(see equation\ref{eq:corr_cross}). For all downstream detectors used in this analysis it is critical to minimize any selection cuts and systematically check their effects on $N_2$.
\end{enumerate}
\subsection{Event Selection}\label{subsec:event-sel}
\begin{figure}[htpb]
    \centering
    \includegraphics[width=\textwidth,height=6cm,keepaspectratio=true]{Figures/SETUP_around_Target.png}
    \caption{
    R3B Setup for the S444 experiment in the target region. TODO: select overview with less numbers/measures...
    }
    \label{fig:setup_target_region}
\end{figure}

For event selection, all three upstream detectors are utilized: the MWPC0, the R3BMUSIC Ionization Chamber, and the start detector. To ensure a clean incoming event selection, the following prerequisites must be met:
\begin{enumerate}
\item \textbf{$^{12}$C identification of incoming projectile by upstream detectors:}\newline
\begin{figure}[htpb]
    \centering
    \includegraphics[width=\textwidth,height=8cm,keepaspectratio=true]{Figures/charge_r3bmusic.png}
    \caption{
    Charge distribution on R3BMUSIC with predefinded calibration parameters with already applied positional cuts on MWPC0 - positioned upstream to the ionisation chamber. The rise beyond Z $\ge$ 7.2 comes from pile-up events. TODO: more explanation needed? 2D plot needed? 
    }
    \label{fig:r3bmusic_charge}
\end{figure}
In the S444 experiment the incoming beam was directly delivered by the SIS18 ring accelerator, which is operated in ultra-high vacuum.The level of contamination is low.(TODO:up to which stage do we have vacuum? Until MW0?)\newline %TODO:up to where is the beam in vacuum?
For the charge identification of the incoming ion the R3BMUSIC ionisation chamber is used which is positioned directly after the MWPC0 at the beam entrance in Cave C, see figure \ref{fig:setup_target_region}. The R3BMUSIC detector measures anode-wise the energy loss of the passing-through ion which in the first order is proportional to the square of its charge ($\Delta E \sim Z^{2}$). Herfore the calibration parameters from the online analysis are used\footnote{These are generic paramteter values used to the detector performance during the experiment phase.}. Figure \ref{fig:r3bmusic_charge} shows the measured charge distribution in R3BMUSIC. To select $Z = 6$ incoming ions the distibution is fitted with a gaussian fit function. All ions with charge within the $\pm 1 \sigma$ range are accepted. Figure \ref{fig:r3bmusic_cuts} summarizes the $\pm 1 \sigma$ cuts on the R3BMUSIC charge for empty/target runs for all beam energies.   
\begin{figure}
\centering
\includegraphics[width=\textwidth,height=8cm,keepaspectratio=true]{Figures/r3bmusic_charge_cuts.png}
\caption{Strict $\pm 1 \sigma$ charge cuts with R3BMUSIC for incoming particle selection. Fixed predefined calibration parameters were used which do not compensate different gain settings between runs. This is in particular the case for the 400 AMeV beam energy runs.}
\label{fig:r3bmusic_cuts}
\end{figure}
\item \textbf{Pileup rejection and TPat selection:}\newline
The overall recoding and merging of the data from various subdetectors is one of the tasks of the Data AcQuisition (DAQ) system. Whether an event is recorded or not depends on the pre-established trigger logic. Various detectors can send out triggers to the main DAQ when certain conditions are given (e.g. CALIFA can be configured to send out a trigger when a hit with more than 20 MeV is recorded in the calorimeter). The different triggers are processed by the trigger logic and summarized as a defined trigger pattern, so called TPat, which is stored in a 16-bit mask for each event. Table \ref{tab:tpats} gives an overview of the trigger logic and the trigger patterns set in the S444 experiment. For this analysis the \textit{"Min. Bias"} trigger is required\footnote{This includes also \textit{"Reaction"} and \textit{"Neutron"} TPat since these patterns contain also \textit{"Min. Bias"} TPat as necessary condition.}.\newline 
\begin{table}[h!]
\centering
\begin{tabular}{||c c c||} 
\hline
Bit Position &TPat Name & Description \\
\hline\hline
0 & Min Bias & Hit in Start detector\\
1 & Reaction & "CalifaOR" -high energy hit in CALIFA \\
2 & Neutron & Hit in NeuLAND \\
3 & p+n & Hit in CALIFA and Neuland \\
8 & Califa & high energy hit in califa - off-spill \\
9 & NeuLAND & Hit in NeuLAND - off-spill \\
\hline\hline
\end{tabular}
\caption{List of TPats set for S444 experiment. As for the selected runs low beam rates ($< 10kHz$) were expected no dead time issues should arise for the in-beam detectors, therefore no downscaling of the \textit{Min. Bias} TPat was deployed.}
\label{tab:tpats}
\end{table}
Since the TPat selection itself does not necessary set any pileup constraints it is important to analyse the signals of the detectors upstream carefully to insure yourself that only events with  one incoming $^{12}$C ion at a time get selected. Herefore events with incoming ions with charge $Z = 6 \pm 1\sigma$ are chosen, as discussed in the previous point. Moreover it is required that both left and right preamplifiers of the start detector have seen a coincident signal within a time-window of 1.391 ns.---TODO: why this time window?-- The overall searching window of the start detector was set to 2 $\mu s$, see figure \ref{fig:start_good_event_sel}. For the MWPC0 which is mounted right at the beam entrance of Cave C no hit multiplicity cuts were applied considering its operating mode, which is designed for charge sharing between the readout pads. 
\begin{figure}
\centering
\includegraphics[width=\textwidth,height=8cm,keepaspectratio=true]{Figures/start_tdiff_exactly2good_hits.png}
\caption{$\Delta t_{right-left}$ between hits in the Start detector for events with exactly one hit on the left and right preamplifier and limiting the time differnce in the range 0.555 ns to 1.946ns.}
\label{fig:start_good_event_sel}
\end{figure}
\item \textbf{Projectile's focus on the active target region:}\newline
To assure that the incoming $^{12}$C ion hits the target it is necessary to select only events where the projectile is focussed to the active target region. Therefore strict cuts on the MWPC0 x and y positon are applied. This was achieved by fitting the x and y distribution of the MWPC0 (without any restrictions on it) by a gaussian function. The selection of focused incoming projectiles was then restricted to events with hits in MWPC0 within the $\pm 1\sigma$ region in the x and y position, see figure \ref{fig:mw0_xy_overview} and \ref{fig:mw0_cuts}.\newline
\begin{figure}
\centering
\includegraphics[width=\textwidth,height=8cm,keepaspectratio=true]{Figures/mw0_xy_summary.png}
\caption{x-y position of incoming ion on MWPC0.}
\label{fig:mw0_xy_overview}
\end{figure}
\begin{figure}
\centering
\includegraphics[width=\textwidth,height=8cm,keepaspectratio=true]{Figures/mwpc0_cutxy.png}
\caption{Overview of $\pm 1\sigma$ cuts in x and y in MWPC0 for empty/target runs. TODO: labelling on the right side is wrong! (should be y!)}
\label{fig:mw0_cuts}
\end{figure}
The MWPC0 x-position and the available projectile angle in the x-y plane from the R3BMUSIC is used to propagate the corresponding x-position on the target location to further check that the selected projectiles hit the target parallel to the z-position (= beam direction) and do only have a minimal incident angle, see figure \ref{fig:x_pos_target}. 
%This can also be seen as a consistency check whether the applied (predefined) calibration parameters are properly set.
\begin{figure}[htpb]
    \centering
    \includegraphics[width=\textwidth,height=8cm,keepaspectratio=true]{Figures/x_proj_beam_mw0_target.png}
    \caption{
   	Propagated x-position on target location from measured x-value on MWPC0 and x-y plane angle measurement from R3BMUSIC. The target area is 3 x 3 cm. In red the selected events with $\pm 1\sigma$ cut in x and y position in MWPC0, in blue all events. TODO: which run is this?
    }
    \label{fig:x_pos_target}
\end{figure}
\end{enumerate}
\subsection{Charge Changing Cross Section Measurement}\label{subsec:cc_cs}
The charge changing cross section refers to a measure of the probability that the incoming projectile will undergo a reaction inside the target that changes its charge. To measure the charge changing cross section it can be referred to formula \ref{eq:corr_cross} where in this case $N_2$ is the number of survived carbon isotopes, i.e. projectiles which did not change their charge state. For this measurement only the data from the double ionisation chamber TWIN Music (see section \ref{sec:ionisation_chambers}) needs to be read out and analyzed.\newline
While for the event seletion before the target the cut conditions can be arbitrarly strict (it will only have an impact to the statistics and the derivated statistical error), cuts on the downstream detectors need to be avoided if at all possible. Too selective cuts on the identification of $N_2$ can distort the measurement.
\subsubsection{TWIN MUSIC Calibration}
For the analyis of data in TWIN MUSIC - different to the upstream detectors, where calibrated data with default calibration parameters is used - the so called \textit{mapped} raw level data is processed. In the mapped level TWIN MUSIC provides following information:
\begin{itemize}
\itemsep0em 
\item \textbf{SectionID:} The detector is a double ionisation chamber and as such divided into four parts (in beam perspective): section 1 - right down; section 2 - right up; section 3 - left down; section 4 - left up. For the S444 experiment only section 1 was operated and accordingly centered on the beam spot.
\item \textbf{AnodeID:} TWIN MUSIC has 16 anodes for energy-loss readout and one refence anode (anodeID = 17).
\item \textbf{Time:} Each hit in each anode gets assigned to a time. Each time has individually no meaning. The drift time (in ns) of the electrons from the ionisation process of the gas by the inflying projectile (or the fragments of it) to the anode is calculated by substracting the individual anode time by the time of the reference anode. The reference anode receives its clean signal from a constant fraction discriminator of the start detector.
\item \textbf{Energy:} Each hit in each anode gets assigned to an energy - except the hit in the reference anode. To reconstruct the charge of the crossing through charged particle anodewise or detectorwise the parmetrization formula Z = [0] + [1]*$\sqrt(E)$ + [2]*E is used. 
\end{itemize}
The calibration of the TWIN energy for each anode is done run-wise. 
%The event selection on the incoming ion is limited to the  \textit{"Min. Bias"}-TPat condition (coincident signal in start detector preamplifiers within 1.391 ns, on-spill).
For the TWIN MUSIC only events where all anodes having exactly one hit (including the reference anode) are chosen. The most prominent peak (Z = 6) was fitted with gaussian function. The calibration was then done by determining the scaling factor for each anode  that shifts the mean of the gaussian fits to the same position, see figure \ref{fig:calibration}. For this analysis the peaks were shifted to $\Delta E = 6$. Since $\Delta E \propto Z^2$ holds, the scaled $\Delta E$ value, even though peaking at 6, is not equate the charge Z = 6.
TODO: describe better the calibration steps!!! this is not good!
\begin{figure}
     \centering
     \begin{subfigure}[t]{0.45\textwidth}
         \centering
         \includegraphics[width=\textwidth]{Figures/twim_mapped_550.png}
         \caption{Uncalibrated raw $\Delta E$ distributions for all 16 anodes for the thick target run, 550 AMeV beam energy.The last six anodes have a slightly different electronics amplification chain.}
         \label{fig:raw_twim}
     \end{subfigure}
     \hfill
     \begin{subfigure}[t]{0.45\textwidth}
         \centering
         \includegraphics[width=\textwidth]{Figures/twim_mapped_550_no_calib.png}
         \caption{Gaussian fit applied to prominent peak and shifted to same position.}
         \label{fig:cal_twim_one}
     \end{subfigure}
     \hfill
        \caption{Fitting procedure in TWIN.TODO: nicer labeling!}
        \label{fig:calibration}
\end{figure}
\subsubsection{TWIN MUSIC Event Selection}
As previously stated cuts on the downstream detectors are avoided. However, events which have hits in one or several anodes in TWIN MUSIC but no signal in the reference anode are discarted as a whole neither contributing to $N_1$ (incoming selected ions) nor to $N_2$ (unreacted ions). If no reference time from start CFD signal is available it is not possible to measure the drift time in the individual anodes which makes it not possible to distinguish between signal and noise hits for multi-hit anode events in TWIN MUSIC. The number of events affected by this cut is in the region of few tens. This is negligible to the number of incoming ions $N_1$ and should not have any dependence whether the projectile reacted or not.
\begin{table}[h!]
\centering
\begin{subtable}[c]{1.\textwidth}
\centering
\begin{tabular}{|c|c|c|c|c|}
\hline
\makecell{\# incoming \\ projectiles $N_1$}& 400 & 550 & 650 & 800 \\
      & MeV/nucleon & MeV/nucleon & MeV/nucleon & MeV/nucleon \\
\hline
Empty & 574279{\footnotesize(*451*)} & 453729{\footnotesize(*34*)} & 522451{\footnotesize(*44*)} & 395451{\footnotesize(*52*)} \\
\hline
thin & 569503{\footnotesize(*422*)} & 476323{\footnotesize(*33*)} & 538037{\footnotesize(*43*)} & 481459{\footnotesize(*36*)} \\
\hline
medium & 606578{\footnotesize(*431*)} & 451137{\footnotesize(*27*)} & 500688{\footnotesize(*40*)} & 345654{\footnotesize(*46*)} \\
\hline
thick & 655762{\footnotesize(*497*)} & 436457{\footnotesize(*30*)} & 530869{\footnotesize(*29*)} & 479679{\footnotesize(*61*)} \\
\hline
\end{tabular}
\caption{Number of clean selected incoming $^{12}$C ions. In brackets number of rejected events because of missing tref in TWIN MUSIC. TODO: change the bracket notation, looks like error number!!}
\label{tab:incoming_ions}
\end{subtable}
\newline
\begin{subtable}[c]{1.\textwidth}
\centering
\begin{tabular}{|c|c|c|c|c|}
\hline
\makecell{\# survived carbon \\ isotopes $N_2$}& 400 & 550 & 650 & 800 \\
      & MeV/nucleon & MeV/nucleon & MeV/nucleon & MeV/nucleon \\
\hline
Empty & 563382{\footnotesize(1.898\%)} & 444618{\footnotesize(2.008\%)} & 511923{\footnotesize(2.015\%)} & 387513{\footnotesize(2.007\%)} \\
\hline
thin & 538245{\footnotesize(5.489\%)} & 449422{\footnotesize(5.648\%)} & 507557{\footnotesize(5.665\%)} & 454099{\footnotesize(5.683\%)} \\
\hline
medium & 552763{\footnotesize(8.872\%)} & 410376{\footnotesize(9.035\%)} & 455159{\footnotesize(9.093\%)} & 314119{\footnotesize(9.123\%)} \\
\hline
thick & 553935{\footnotesize(15.528\%)} & 368004{\footnotesize(15.684\%)} & 446115{\footnotesize(15.965\%)} & 402696{\footnotesize(16.049\%)} \\
\hline
\end{tabular}
\caption{Number of survived carbon isotopes after the target identified via 2D gaussian fit with borders within 3.5$\sigma$ cut. In brackets the precentage of projectiles with a charge state of Z < 6 after the target.}
\label{tab:survived_ions}
\end{subtable}
\caption{Numbers of incoming projectiles $N_1$ and survived carbon isotopes $N_2$ for all energy and target runs.}
\label{tab:overview_nr_cccs}
\end{table}
\subsubsection{Carbon Identification}\label{subsec:carbon_id}
The identification of carbon isotopes in TWIN is done by reconstructing fragments with charge Z = 6 from 2D plots where coincident mean energy losses $\Delta E$ for different anode combinations are plotted. Since the TWIN MUSIC is multi-hit capable various strategies were developed to deal with multi-hit events, i.e. when having anodes with multiple hits, decide which hit originates from the final state products from the reaction and which from background and noise.\newline
The default strategy is to use the time information of each hit for selection. It has to be remarked that for the S444 experiment the TWIN MUSIC was read out by two independent MDPP modules\cite{MDPP-16}. The signals from the first reference anode and the first eight upstream anodes were sent to module 1, the ones from the last eight downstream anodes and the second reference anode were forwarded to module 2. For the first eight upstream anodes the drift time is calculated by substracting the hit time in each anode by the reference time from the first reference anode and for the last eight downstream anodes accordingly the second reference andode was used.\newline
The time based selection algorithm for multi-hit anodes works as follows:\newline
\begin{enumerate}
\itemsep0em
\item Get the mean drift time for the eight upstream anodes($t_{mean\_up}$) and the eight downstream anodes($t_{mean\_down}$)\footnote{For the case all eight downstream anodes have multiple hits, set $t_{mean\_down} = t_{mean\_up}$ and vice versa}. Anodes with multiple hits do not contribute to this calculation.
\item If there are anodes with muliple hits compare the hit time with the accorging mean drift time ($t_{mean\_up}$ for any of the eight upstream anodes, $t_{mean\_down}$ for any of the eight downstream anodes). Calculate herefore the absolute difference between mean drift time and each hit time:
\begin{equation}
\Delta t = | \bar{t} - {t^i}_{drift}|; \, \text{i = anodeID (1-16) with} \quad \bar{t} = 
\begin{cases}
t_{mean\_up} & \text{for i} \leq 8\\
t_{mean\_down} & \text{for i} \geq 9 \\
\end{cases}
\end{equation}
\item For each anodes with muliple hits select the hit with lowest drift time differnce to the mean drift time.
\end{enumerate}
\begin{figure}[htpb]
    \centering
    \includegraphics[width=\textwidth,height=8cm,keepaspectratio=true]{Figures/charge_cut_with_out_borders.png}
    \caption{
    Two dimensional gaussian fit with 3.5 $\sigma$ cut  on identified carbon isotopes in TWIN MUSIC. The horizontal and vertical side bars contain events where either the eight upstream anodes or downstream anodes have no hit entry. The cluster at scaled $\Delta E$ loss $\approx$ 4.5 corresponds to boron isotopes (Z=5). The clusters for Z=4 (Be) and Z=3(Li) overlap.
    }
    \label{fig:twin_2d_gaus_cut}
\end{figure}

After having selected the appropriate hit for single and multi-hit anodes the mean value for the pre-calibrated $\Delta E$ loss, see figure \ref{fig:cal_twim_one}, for the eight upstream and accordingly for the eight downstream anodes is determined. Finally, to select the number of survived carbon isotopes the mean $\Delta E$ of the eight upstream anodes versus the mean $\Delta E$ of the eight downstream anodes is plotted. To retrieve the number of survived carbon isotopes following two-dimensional gaussian fit is applied on the 2D plot on the charge Z = 6 blob, see figure \ref{fig:twin_2d_gaus_cut}:
\begin{equation}
f(x) = A e^{-\frac{1}{2}((\frac{x - \bar{x}}{\sigma_{x}})^2 +(\frac{y - \bar{y}}{\sigma_{y}})^2)}
\label{eq:gauss_fit}
\end{equation}
where x is the mean rescaled energy loss of the first upstream andodes and y the according eight downstream anodes. The number of survived carbon isotopes is given by the integral of events within the 2D gaussian fit. Since the anodes were read out by two independent MDPP modules with slightly different thresholds also events along the histogram axes with no hit entry in either the upstream anodes or downstream anodes are analyzed. For those events a one dimensional gaussian cut is applied using the parameters from equation \ref{eq:gauss_fit} (see horizontal and vertical bars in figure \ref{fig:twin_2d_gaus_cut}).\newline
To get the charge changing cross section values equation \ref{eq:corr_cross} has to be applied where both the number of survived  carbon isotopes for target run and empty run are determined via the 2D gaussian fit as in figure \ref{fig:twin_2d_gaus_cut}. The number of target particles, are defined by the target thickness and its density and are listed in section \ref{section:transmission_method}. 
%\begin{equation}
%\begin{split}
%&N_t = \rho \cdot N_A \cdot n \cdot d\\ 
%&\rho = \text{density [$g/cm^3$]} = 1.851 g/cm^3,\text{taken from \cite{ponnath2023precise}}\\
%&N_A = \text{Avogadro constant} =  6.02214076 \cdot 10^{23} mol^{-1}\\
%&n = \text{amount of substance [mol/g]} = 1./12.011 \; mol/g\\
%&d = \text{target thickness[cm]}
%\end{split}
%\end{equation}
%Three carbon targets with different thicknesses are given:
%\begin{enumerate}
%\itemsep0em
%\item thin target: d = 0.5451 cm
%\item medium target: d = 1.0793 cm
%\item thick target: d = 2.1928 cm 
%\end{enumerate}
\begin{figure}[htpb]
    \centering
    \includegraphics[width=\textwidth,height=8cm,keepaspectratio=true]{Figures/cccs_with_out_border_3_5_sigma.png}
    \caption{
    Charge changing cross section without geometry corrections. The red data points result from considering also events with only hits in the upstream or downstream anodes, the blue data points don't take these events in consideration. 
     }
    \label{fig:cccs_with_out_border_3_5}
\end{figure}
The resulting charge changing cross sections are summarized in figure \ref{fig:cccs_with_out_border_3_5} once with consideration of the vertical/horizontal bars in figure \ref{fig:twin_2d_gaus_cut} and once without. Hits within the 3.5 $\sigma$ gaussian fit are identified as carbon isotopes.
To get the optimal $\sigma$ cut on the two dimensional gaussian fit on the energy losses of the upstream anodes versus downstream anodes the charge changing cross section for all targets and all energies was systematically measured for $\sigma$-cuts in the range of 1 to 5 $\sigma$, see figure \ref{fig:cccs_vs_sigma_cut}. In the region $\thicksim$ 3.5$\sigma$ the variation of the cross section is minimal.
\begin{figure}[htpb]
    \centering
    \includegraphics[width=\textwidth,height=8cm,keepaspectratio=true]{Figures/cccs_vs_sigma_cut.png}
    \caption{
    Measured charge changing cross sections according to the $\sigma$ cut applied on the figure \ref{fig:twin_2d_gaus_cut} (with borders) for the different target thicknesses and beam energies.
     }
    \label{fig:cccs_vs_sigma_cut}
\end{figure}
Another method to assert the number survived carbon ions is to apply a diagonal cut on the 2D $\Delta E$ histogram. To set the slope and offset of the diagonal cut line firstly the two dimensional gaussian fit is applied, same as for the previous method. Then the intersection point between the 3.5 $\sigma$ ellipse and the identity line ($\Delta E$ upstream anodes = $\Delta E$ downstream anodes) is found. Through this point, perpendicular to the identity line, the diagonal line is drawn. Everything above the diagonal line is considered as survived carbon ions. Moreover the borders are considered within the 3.5 $\sigma$ cut, see figure \ref{fig:diagonal_cut_twim}. 
\begin{figure}[htpb]
    \centering
    \includegraphics[width=\textwidth,height=8cm,keepaspectratio=true]{Figures/charge_loss_550_thick.png}
    \caption{
    Diagonal cut on identified carbon isotopes along the gaussian 3.5 $\sigma$ cut with borders. All hits above the diagonal line are counted as carbon isotopes. Histogram from thick target run, 550 AMeV beam energy.
     }
    \label{fig:diagonal_cut_twim}
\end{figure}
The effects of the two different methods used for the identification of the carbon isotopes for the charge changing cross section is summarized in figure \ref{fig:cccs_gaus_vs_diag}. The differences in the measured cross sections are within the margin of error herefore both methods are comparable, as expected.
\begin{figure}[htpb]
    \centering
    \includegraphics[width=\textwidth,height=8cm,keepaspectratio=true]{Figures/cccs_gauss_diag_comp_3_5_sigma.png}
    \caption{
    Comparison of charge changing cross section measured via 2D gaussian fit cut and diagonal cut. The differnces are within the margin of error.
     }
    \label{fig:cccs_gaus_vs_diag}
\end{figure}
To check wether single anodes or groups of anodes are malfunctioning the charge changing cross section measurement was repeated using only certain anodes for the charge identification:
\begin{enumerate}[label=\alph*)]
\itemsep0em
\item anodes 2-8 versus anodes 9-15 (omitting first and last anode)
\item anodes 1-4 versus anodes 5-8 (upstream anodes)
\item anodes 5-8 versus anodes 9-12 (central anodes)
\item anodes 9-12 versus anodes 13-16 (downstream anodes)
\end{enumerate}
The results from the measurement are summarized in figure \ref{fig:cccs_gaus_diff_sections}. The difference between the default gaussian fit method (with 3.5$\sigma$ cut and considering the borders) considering all 16 anodes and applying the same method but omitting the first and last anode is minimal over all four beam energies. When selecting only 8 out of 16 anodes instead the cross sections are systematically lower when going to high beam energies. The energy loss inside the TWIN MUSIC decreases with higher beam intensities, according to the Bethe-Bloch formula:

%\[
\begin{equation}
-\frac{dE}{dx} = K z^2 \frac{Z}{A} \frac{1}{\beta^2} \left( \frac{1}{2} \ln \frac{2 m_e c^2 \beta^2 \gamma^2 T_{\text{max}}}{I^2} - \beta^2 - \frac{\delta(\beta \gamma)}{2} \right)
\end{equation}
%\]

\text{where:}
\begin{align*}
K &= 4 \pi N_A r_e^2 m_e c^2 \approx 0.307 \, \text{MeV} \, \text{cm}^2 \, \text{g}^{-1}, \\
z &= \text{charge of the incident particle (in elementary charge units)}, \\
Z &= \text{atomic number of the target material}, \\
A &= \text{atomic mass of the target material}, \\
\beta &= \frac{v}{c} = \text{velocity of the particle relative to the speed of light}, \\
\gamma &= \frac{1}{\sqrt{1 - \beta^2}} = \text{Lorentz factor}, \\
T_{\text{max}} &= \text{maximum kinetic energy transferable to an electron in one collision}, \\
I &= \text{mean excitation potential of the target material}, \\
\delta(\beta \gamma) &= \text{density effect correction}.
\end{align*}
The behaviour of dE/dx for small $\beta$- values are dominated by the 1/$\beta^2$ term. The decrease of deposited energy for larger beam energies has as consequence a lower relative resolution in the two dimensional $\Delta E$ loss histogram (see figure \ref{fig:twin_2d_gaus_cut}) reflecting the poissonian distribution properties. In addition reducing the number of readout anodes by a factor two degrades the resolution by a factor $\sqrt{2}$\footnote{It can be assumed a similar $\Delta$ E distribution for all anodes. Hence the central limit theorem can be applied where $\sigma = \frac{\sigma_{anode}}{\sqrt(n)}$ with $n = number of anodes$.}. This has as consequence that the ellipsis with 3.5$\sigma$ cut incorporates a non negligible amount of boron isotopes which are counted as survived carbon isotopes which in turn reduces the measured charge chaning cross section.\newline
\begin{figure}[htpb]
    \centering
    \includegraphics[width=\textwidth,height=8cm,keepaspectratio=true]{Figures/cccs_various_sections_2dgauss_border.png}
    \caption{
   	Measurement of charge changing cross sections using different anode sections to make the two dimensional gaussian fit on the identified carbon isotopes. Red: using all 16 anodes. Blue: the various combinations. 
     }
    \label{fig:cccs_gaus_diff_sections}
\end{figure}
While in the above measurements a time based secelction algorithm for multi-hit anodes was used also an energy based selection was tested. This algorithm selects for multi-hit anodes the hit with the highest energy as physical hit and discarts all others, as they are considered as background/noise. Figure \ref{fig:cccs_gaus_time_vs_energy} compares the time based method versus the energy based method. In both cases a two dimensional gaussian 3.5$\sigma$ cut is applied as in figure \ref{fig:twin_2d_gaus_cut} and the borders are counted as well. The difference in the outcome is negligible. This can be explained since noise or background signal should be both uncorrelated to the event time and at a low energy level and are therefore filtered by both algorithms.
\begin{figure}[htpb]
    \centering
    \includegraphics[width=\textwidth,height=8cm,keepaspectratio=true]{Figures/cccs_time_vs_engergy_sorted.png}
    \caption{
    Comparison of charge changing cross section measurements when using time sorting algorithm(red) and energy sorting algorithm(blue) for multi-hit anodes.
     }
    \label{fig:cccs_gaus_time_vs_energy}
\end{figure}
The final charge changing cross section measurements with 2D gaussian fit applying a 3.5 $\sigma$ cut and including the borders of the histogram are summarized in figure \ref{fig:cccs_gaus_with_errors}. At this stage also the statistical errors are incorporated. 
%TODO: Description of statistical gaussian error propagation (refer to L.Ponnath thesis).
\begin{figure}[htpb]
    \centering
    \includegraphics[width=\textwidth,height=8cm,keepaspectratio=true]{Figures/cccs_with_stat_errors.png}
    \caption{
    Measurement of charge changing cross sections using all 16 anodes of the TWIN MUSIC applying the 2D gaussian fit and considering the borders as in figure \ref{fig:twin_2d_gaus_cut}.
     }
    \label{fig:cccs_gaus_with_errors}
\end{figure}

\newpage
%note that thickness calc from Lukas --TODO
%-> deltaE possonian distributed, according to central limit theorem, the sum too DONE
%-> therefore plot Esum of first 8 anodes vs Esum of last 8 anodes DONE
%-> two modules were used for readout DONE
%-> For selecting charge = 6 make gaussian fit on Esum up down DONE
%-> Compare charge = 6 cuts: DONE
%	-->show gaussian 2D cut, for all sigmas (with and whithout borders)
%	-->show gaussian diagonal cut for all sigmas (with and without borders)
%	-->show effect when discarting first/last, first two/last, first three/last and only using last 8 or last 6 anodes with fixed sigma cut
%	
\subsection{Geometric Corrections}\label{sec:geo_corr}
For the S444 experiment only section 1 (right down) of the TWIN MUSIC, which was centered on the beam spot, was operated. As consequence full geometric efficiency could not be assumed. To visualize the restricted geometic efficiency of the TWIN MUSIC the position in x and y (perpendicular plane to the beam direction) on the MWPC1 in front of the ionisation chamber was plotted, once without any conditions on the TWIN MUSIC and once with the condition of having identified a carbon isotope (with the 2D gauss-fit method as described in chapter \ref{subsec:carbon_id}), see figure \ref{fig:mw1_xy}. The large active surface area of 200 x 200 mm$^{2}$ of the MWPC1 affirms that all the carbon fragments are detected\footnote{This statement does not hold for light fragments as protons or deuterons. Their deflection angle exceeds the geometric acceptance of the MWPC1.} whereas the TWIN MUSIC behind it, with an active survace of 55 x 110 mm$^{2}$ (section 1), is not sensitive to the scattered fragments with large deflection angle.\newline
\begin{figure}[htpb]
    \centering
    \includegraphics[width=\textwidth,height=8cm,keepaspectratio=true]{Figures/geo_corr_examples.png}
    \caption{
    Distribution in x and y on MWPC1 for different energies with and without target. TODO: subplot b) wrong naming,it's the empty run 400 AMeV. 
     }
    \label{fig:mw1_xy}
\end{figure}

The efficiency loss depends on the target thickness - for thicker targets the geometric distribution of the fragments is broader and therefore the efficiency loss larger - and the beam energies - for larger beam energies the scattering angles decrease due to the boost effects, the efficiency loss gets smaller. This means, that the efficiency loss needs to be compensated runwise. These efficiency effects can be observed in figure \ref{fig:mw1_xy}.\newline
To compensate correctly for the geometric efficiency it has to be considered that for the charge changing cross section measurement only the carbon isotopes after the target are counted in the TWIN MUSIC. Therefore the correction should only be applied to the carbon isotopes (Z = 6) on the x-y distribution on the MWPC1, see figure \ref{fig:mw1_xy}. The geometric efficiency correction is done graphically on the x-y distribution of the MWPC1 for carbon isotopes by following procedure:
\begin{enumerate}
\itemsep0em
\item \textbf{Correction for the x-position distribution:}
%\item TODO: insert x-Distribution in MW1 and the fitted functions
\begin{enumerate}
\item First fit x-distribution with double-gaussian function with five free parameters and common mean value $\mu_x$
\begin{equation}
f(x) = A \cdot exp\left(-\frac{(x-\mu_x)^2}{a^2} \right) + B \cdot exp\left(-\frac{(x-\mu_x)^2}{b^2} \right)
\end{equation}
\item Fit again within range $\mu_x \pm \epsilon_x$.The parameter $\epsilon_x$ is fixed by educated guess, TODO.As $\mu_x$ take the value from the fit in the previous step. A fit for the central region of the x-distribution is obtained,\newline
$f(x)_{central}(A_{central},a_{central},B_{central},b_{central},\mu_{central})$.
\item The obtained fit function $f(x)_{central}$ is then used to compare with the data distribution($f(x)_{data}$) in the border regions [-100,$\mu_{central}-\epsilon_x$] and [$\mu_{central} + \epsilon_x$,100]. Since only the left border region (low x-positions) is affected by the limited geometric acceptance, the right border region can be used for correction:
\begin{equation}
\Delta_{xcorr} = \int_{\mu_{central} + \epsilon_x}^{100} f(x)_{data} - f(x)_{central}\; - \;\int_{-100}^{\mu_{central} - \epsilon_x} f(x)_{data} - f(x)_{central} 
\end{equation}
\end{enumerate}
\item \textbf{Correction for the y-position distribution:}
%\item TODO: insert y-Distribution in MW1 and the fitted functions
\begin{enumerate}
\item First fit y-distribution with double-gaussian function with five free parameters and common mean value $\mu_y$
\begin{equation}
f(y)_{fit} = C \cdot exp\left(-\frac{(y-\mu_y)^2}{c^2} \right) + D \cdot exp\left(-\frac{(y-\mu_y)^2}{d^2} \right)
\end{equation}
\item The obtained fit function $f(y)$ is then used to compare the data distribution($f(y)_{data}$) in the border regions [-100,$\mu_y-\epsilon_y$] and [$\mu_y + \epsilon_y$,100]. The parameter $\epsilon_y$ is fixed by educated guess, TODO. As $\mu_y$ take the value from the fit in the previous step. Same as for the x-correction both border regions are compared. The high border region (high y-positions) affected by the limited geometric acceptance while the low border region (low y-positions) has full geometric acceptance.
\begin{equation}
\Delta_{ycorr} = \int_{-100}^{\mu_{central} - \epsilon_y} f(y)_{data} - f(y)_{fit} \; - \; \int_{\mu_x + \epsilon_y}^{100} f(y)_{data} - f(y)_{fit} 
\end{equation}
\end{enumerate}
\item To correct the number of survived carbon isotopes $N_2 = N_{carbon}$ both corrections in x and y are applied\footnote{Under the assumption that x and y are uncorrelated where the x-y distribution on the MWPC1 is given by $f(x,y) = f(x)\cdot f(y)$.}
\begin{equation}
N_2^{corr} = N_2 +\frac{\Delta_{xcorr}+\Delta_{ycorr}}{N_2}
\end{equation}
\end{enumerate}
\begin{figure}[htpb]
    \centering
    \includegraphics[width=\textwidth,height=8cm,keepaspectratio=true]{Figures/corig_geo_corr_factor.png}
    \caption{
    Geometric correction factors from limited geometric efficiency of TWIN Music. 
     }
    \label{fig:geo_corr_twim}
\end{figure}
Figure \ref{fig:geo_corr_twim} summarizes the geometric correction factors $\epsilon_{geo\textunderscore corr}$ obtained from the graphical reconstruction of missed TWIN Music events as described above. The correction factor is subsequently applied to the charge-changing cross-section, resulting in the final corrected charge-changing cross-section:
\begin{equation}
\sigma_{geo\textunderscore corr} = -\frac{1}{N_t} ln(\frac{N_{1}^E}{N_{2}^E}\frac{N_2}{N_1}\cdot \epsilon_{geo\textunderscore corr}) = -\frac{1}{N_t} \left( ln(\frac{N_{1}^E}{N_{2}^E}\frac{N_2}{N_1}) + ln(\epsilon_{geo\textunderscore corr})\right)
\label{eq:geo_corr_cccs}
\end{equation}
\begin{figure}[htpb]
    \centering
    \includegraphics[width=\textwidth,height=8cm,keepaspectratio=true]{Figures/charge_changing_geo_corrected_cross_section.png}
    \caption{
    Charge changing cross section with applied geometry corrections. In gray: charge changing cross section measurements before applying corrections, as in figure \ref{fig:cccs_gaus_diff_sections}. 
     }
    \label{fig:geo_corr_cross_sec}
\end{figure}
Figure \ref{fig:geo_corr_cross_sec} shows the measured charge changing cross section once without considering the the limited geometric acceptance of the TWIM Music and once applying the geometric correction factors as presented in equation \ref{eq:geo_corr_cccs}. As expected significantly affected by the  geometric correction are runs with 400 and 550 AMeV beam energy whereas at beam energies of 650 and 800 AMeV the effect is exceptionally small since at high beam energies  the fragments after the target preceive a strong boost effect in beam direction which constrains the distribution in the x-y-plane.

\subsection{Isotope Correction - Total Interaction Cross Section}
The general formulation for the calculation of cross sections in equation \ref{eq:corr_cross} can be used to determine the cross section for specific channels depending on the definition of $N_2$. In the previous subsection \ref{subsec:cc_cs} where the charge changing cross section was measured, $N_2$ had to be sensitive to the charge of the outgoing fragment. Therefore $N_2$  was  defined as the number of survived carbon isotopes, i.e. $N_2 = N_2^{Z=6}$. For the measurement of the total interaction cross section $N_2$ has to be sensitive to both proton and neutron number of the fragments. $N_2$ is herefore restricted to the number of survived  $^{12}$C isotopes, i.e. $N_2 = N_2^{^{12}C}$. Since $N_2^{^{12}C}$ is a subset of $N_2^{Z=6}$, $N_2^{^{12}C}$ can be determined by identifying and disentangling the number of survived $^{12}$C isotopes from the set of events with carbon isotopes $N_2^{Z=6}$. For that reason the positional correlations on the x-coordinate of MWPC2 (upstream to GLAD) and the MWPC3 (downstream to GLAD) are exploited. The GLAD magnet, which acts as a mass spectrometer, deflects the passing through fragment. Depending on its proton to neutron ratio the fragment is deflected more or less, described by the formula for magnetic rigidity:\newline
\begin{equation}
B\cdot \rho = \frac{\gamma \cdot m \cdot v}{q}
\end{equation}
\text{where:}
\begin{align*}
    B &\quad \text{is the strength of the magnetic field,} \\
    \rho &\quad \text{is the radius of curvature of the particle's trajectory,} \\
    \gamma &\quad \text{is the Lorentz factor,} \\
    v &\quad \text{is the velocity of the particle,} \\
    m &\quad \text{is its mass,} \\
    q &\quad \text{is its charge.}
\end{align*}
Figure \ref{fig:x_mw23_default} shows the x distribution on MWPC2 versus MWPC3 for the thick target run at a beam energy of 400 AMeV. The main diagonal line corresponds to the $^{12}$C isotopes. From all isotopes they have the largest mass to proton ratio (n+p/p) and are therefore less deflected by the magnetic field of GLAD. The less prominent line below corresponds to $^{11}$C isotopes and on the lower edge few events with $^{10}$C are visible. To identify the number of survived $^{12}$C fragments a graphical selection of the $^{12}$C isotopes - the main diagonal line - is applied.\newline
\textbf{Graphical Selection Algorithm:}
\begin{itemize}
\itemsep0em
\item Fit the main diagonal line, which corresponds to the $^{12}$C isotopes, with a linear fit function $f(x_{mw2}) = a \cdot x_{mw2} + b$.
\item To get the most accurate intersection line between $^{12}$C isotopes and all lighter carbon isotopes the linear fit function from the previous step is taken as starting point. Iteratively the offset value $b$ is reduced by $b_i = b_{i-1} -1$. For all iteration steps the ratio $r_{^{12}C(i)}$ of hits below the linear fit function and the total number of hits in the histogram is calculated. 
\item The derivative $\diff{r_{^{12}C}(i)}{b_i}$ is calculated.  
\item Finally the offset $b_i$ with the largest value for $\diff{r_{^{12}C}(i)}{b_i}$ is selected as cutting line between $^{12}$C isotopes and $^{11}$C/$^{10}$C isotopes\footnote{For empty target runs the offset $b$ is manually selected}.
\end{itemize}
The ratio ${r_{^{12}C}}$ is unaffected by detector efficiencies of MWPC2 and MWPC3\footnote{Under the assumption of constant efficiency over $x_{mw2}$ and $x_{mw3}$} and therefore the ratio ${r_{^{12}C}}$ can be applied directly as  isotope correction to calculate the total interaction cross section:
%\begin{equation}
\begin{flalign*}
\sigma_I &= \sigma_{geo\textunderscore corr} + \sigma_{iso} &\\
\text{With the isotope correction cross section $\sigma_{iso}$:} & &\\
\sigma_{iso} &= -\frac{1}{N_t} ln(r_{^{12}C}) & 
\end{flalign*}
%\end{equation}\label{eq:tot_inter_cs}
To calculate the isotope correction cross section six different methods were employed and compared with each other:
\begin{itemize}
\itemsep0em
\item \textbf{MWPC2 and MWPC3 hit-level data:}\newline
For all MWPCs the standard  \textit{cal-to-hit} step  sorts the calibrated hits in the detector according to the calibrated charge deposited in the pads. The final position (in mm) is determined by selecting the hit with the highest charge deposition $Q_{max}$ and its left ($Q_L$) and right neighbour ($Q_R$) pads\footnote{In case no charge deposition in one or both neighbours the charge value is set to 1 respectively.}. These charge and position values are inserted in the "hyperbolic squared secant" function \cite{lau1995optimization} with the following charge distribution function:
\[
Q(x) = \frac{a_1}{\cosh^2\left(\frac{\pi (x - a_2)}{a_3}\right)}
\]
\vspace{-0.5em} % Reduces the space between equations
where \(a_1\) is the amplitude of the distribution $Q_{max}$, \(a_2\) its centroid, and \(a_3\) derives as follows:
\[
a_3 = \frac{\pi \omega}{\cosh^{-1}\left(0.5 \times \left(\sqrt{\frac{Q_{\text{max}}}{Q_L}} + \sqrt{\frac{Q_{\text{max}}}{Q_R}}\right)\right)}
\]
\vspace{-0.5em} % Reduces the space between equations
\(\omega\) being the width of the pads. The centroid of the distribution, which is used as final hit position in the \textit{hit-}data level, can be deduced from:
\vspace{-0.5em} % Reduces the space between equations
\[
a_2 = \frac{a_3}{\pi} \times \tanh^{-1}\left(\frac{\sqrt{\frac{Q_{\text{max}}}{Q_L}} - \sqrt{\frac{Q_{\text{max}}}{Q_R}}}{2 \sinh\left(\frac{\pi \omega}{a_3}\right)}\right)
\]
Figure \ref{fig:hyp_function} shows the "hyperbolic squared secant" function with the inserted values for $Q_{max}$, $Q_R$ and $Q_L$. 
\begin{figure}[htpb]
    \centering
    \includegraphics[width=\textwidth,height=8cm,keepaspectratio=true]{Figures/hyperbolic_squared_secant_function.png}
    \caption{
   	 Figure taken from \cite{martin2021fission}, with w being the with of the cathode pads of the MWPC and $a_2$ the final position value of the hit determined by the hyperbolic squared secant function (in red). In black the measured charge deposition distribution in the MWPC. 
     }
    \label{fig:hyp_function}
\end{figure}
The "hyperbolic squared secant" function is used to determine the x hit position as well as the y hit position for all MWPCs. Figure \ref{fig:x_mw23_default} shows the $x_{mw2}$ versus $x_{mw3}$ distribution of carbon isotopes for the 400 AMeV run with the thick target. The two correlated lines corresponding to the $^{12}$C and $^{11}$C isotopes can clearly be distinguished. The vertical line can be interpreted as amount of events where the incoming centered carbon fragment gets scattered by air or the detector material in place between MWPC2 and MWPC3. The horizontal wide spread line has no physical interpretation and can rather be exlained by the \textit{cal-to-hit} step in MWPC2: For events where there is not a spatially constrained hit cluster but sparse hits the hyperbolic squared secant function may pick the wrong $Q_{max}$ and therefore wrongly reconstructs the x position in MWPC2. 
\begin{figure}[htpb]
    \centering
    \includegraphics[width=\textwidth,height=8cm,keepaspectratio=true]{Figures/mw23_default.png}
    \caption{
   	 Distribution of x in MWPC2 and MWPC3 for the 400 AMeV run with thick target. The green line corresponds to the intersection line between $^{12}$C and $^{11}$C/$^{10}$C isotopes fixed on the graphical selection algorithm.
     }
    \label{fig:x_mw23_default}
\end{figure}
\item \textbf{MWPC2 and MWPC3 data with own "hit-clustering" level:}\newline 
To overcome the issue with potentially wrong x-position reconstruction in the MWPCs the event selection on MWPC2 and MWPC3 was restricted to events where both MWPC2 and MWPC3 have only one spatially constrained cluster(see figure \ref{fig:own_clustering}) to avoid ambiguities in the position determination. This method strongly retains uncorrelated hits in MWPC2 and MWPC3.\newline
\begin{figure}
\floatbox[{\capbeside\thisfloatsetup{capbesideposition={left,top},capbesidewidth=6cm}}]{figure}[\FBwidth]
{\caption{Resticted event selection for MWPC2 and MWPC3: only events with one single coherent (i.e. without any holes) cluster are accepted.}\label{fig:own_clustering}}
{\includegraphics[width=7cm]{Figures/own_clustering_mwpcs.png}}
\end{figure}
Figure \ref{fig:mw23_own_clustering} shows the distribution of x in MWPC2 and MWPC3 using the own hit-clustering reconstruction. This reconstruction method removes the uncorrelated horizontal line which was observed in figure \ref{fig:x_mw23_default}. However the statistics are reduced by $\approx 35\%$\footnote{Number of entries in the 2D plot for default reconstruction method: 533816, for the own hit clustering reconstruction: 346315 for the 400 AMeV run with thick target.}.
\begin{figure}[htpb]
    \centering
    \includegraphics[width=\textwidth,height=8cm,keepaspectratio=true]{Figures/mw23_own_clustering.png}
    \caption{
   	 Distribution of x in MWPC2 and MWPC3 using the own hit clustering reconstruction. Thick target run, beam energy 400 AMeV. The green line corresponds to the intersection line between $^{12}$C and $^{11}$C/$^{10}$C isotopes fixed on the graphical selection algorithm.
     }
    \label{fig:mw23_own_clustering}
\end{figure}

\item \textbf{MWPC1 and MWPC3 hit-level data:}\newline
To get the ratio ${r_{^{12}C}}$ it is necessary to correlate the x positon before and after the GLAD magnet. This task can be completed by MWPC3 with respect to MWPC2 or MWPC1. Since the MWPC1 is upstream to MWPC2 the positional distribution of the carbon fragments narrower which as consequence makes it more difficult to disentangle $^{12}$C and $^{11}$C/$^{10}$C isotopes, see figure \ref{fig:mw13_standard_cluster}. Moreover MWPC1 had two noisy pads which distorts the distribution when using the standard  \textit{cal-to-hit} step to get the position value, see figure \ref{fig:mw13_standard_cluster}.
\begin{figure}[htpb]
    \centering
    \includegraphics[width=\textwidth,height=8cm,keepaspectratio=true]{Figures/mw13_standard_cluster.png}
    \caption{
   	 Distribution of x in MWPC1 and MWPC3. Thick target run, beam energy 400 AMeV. The two vertical lines stem from two noisy pads in MWPC1. 
     }
    \label{fig:mw13_standard_cluster}
\end{figure}
\item \textbf{MWPC1 and MWPC3 data with own "hit-clustering" level:}\newline
Again, to overcome the issue with potentially wrong x-position reconstruction in the MWPCs the own hit clustering reconstruction method, as described above, was applied to MWPC1 and MWPC3 resulting in the 2D plot \ref{fig:mw13_own_cluster}.
\begin{figure}[htpb]
    \centering
    \includegraphics[width=\textwidth,height=8cm,keepaspectratio=true]{Figures/mw13_own_cluster.png}
    \caption{
   	 Distribution of x in MWPC1 and MWPC3 using the own hit clustering reconstruction. Thick target run, beam energy 400 AMeV. 
     }
    \label{fig:mw13_own_cluster}
\end{figure}

\item \textbf{MWPC2 and MWPC3 with own clustering, quadrant selection in MWPC2:}\footnote{I did the quadrant selection also in MWPC2, outcome really similar, TODO: add to appendix...}\newline
The limited geometric acceptance of TWIN MUSIC, described in section \ref{sec:geo_corr}, affects the isotope correction too. The x distribution on the MWPC1/2/3 is cut off on the lower end and the y distribution on the higher end, see figure \ref{fig:mw1_xy}. The $^{11}$C/$^{10}$C isotopes are expected to have a broader x and y distribution. Missing the lower and higher edges in the x and y distribution respectively distorts the ${r_{^{12}C}}$ ratio towards higher values. This results in a lower cross section contribution of the isotope correction, especially for the low energy runs. To correct for this the x-y distribution in MWPC1 was split into four quadrants, see figure \ref{fig:mw1_xy_quadrants}. The intersection lines were derived by the mean of the gaussian fit of the x and y distribution. The cross section contribution of the isotope correction was measured for all four quadrants.
\begin{figure}[htpb]
    \centering
    \includegraphics[width=\textwidth,height=8cm,keepaspectratio=true]{Figures/quadrants_mw1_xy_400_thick.png}
    \caption{
   	 Distribution of x and y in MWPC1 split up in four quadrants.Thick target run, beam energy 400 AMeV. 
     }
    \label{fig:mw1_xy_quadrants}
\end{figure}
\end{itemize}
\newpage

\subsubsection{Results for Isotope Correction Methods}
\begin{itemize}
\item \textbf{MWPC2 and MWPC3 hit-level data:}\newline
\begin{figure}[h!]
    \centering
    \includegraphics[width=\textwidth,height=8cm,keepaspectratio=true]{Figures/iso_corr_contr_default_cluster_mw23.png}
    \caption{
	Isotope correction contribution to the total interaction cross section using standard hit level data in MWPC2 and MWPC3.
     }
    \label{fig:iso_corr_mw23_default}
\end{figure}

\item \textbf{MWPC2 and MWPC3 data with own "hit-clustering" level:}\newline 
\begin{figure}[h!]
    \centering
    \includegraphics[width=\textwidth,height=8cm,keepaspectratio=true]{Figures/iso_corr_contr_own_cluster_mw23.png}
    \caption{
	Isotope correction contribution to the total interaction cross section using own  hit clustering in MWPC2 and MWPC3.
     }
    \label{fig:iso_corr_mw23_own_cluster}
\end{figure}

\item \textbf{MWPC1 and MWPC3 hit-level data:}\newline
\begin{figure}[h!]
    \centering
    \includegraphics[width=\textwidth,height=8cm,keepaspectratio=true]{Figures/iso_corr_contr_default_cluster_mw13.png}
    \caption{
	Isotope correction contribution to the total interaction cross section using standard hit leved data  in MWPC1 and MWPC3.
     }
    \label{fig:iso_corr_mw13_default}
\end{figure}

\item \textbf{MWPC1 and MWPC3 data with own "hit-clustering" level:}\newline
\begin{figure}[h!]
    \centering
    \includegraphics[width=\textwidth,height=8cm,keepaspectratio=true]{Figures/iso_corr_contr_own_cluster_mw13.png}
    \caption{
	Isotope correction contribution to the total interaction cross section using own  hit clustering in MWPC1 and MWPC3.
     }
    \label{fig:iso_corr_mw13_own_cluster}
\end{figure}

\item \textbf{MWPC2 and MWPC3 with own clustering, quadrant selection in MWPC1:}\newline
\begin{figure}[h!]
    \centering
    \includegraphics[width=\textwidth,height=8cm,keepaspectratio=true]{Figures/iso_corr_contr_own_cluster_mw23_quadmw1.png}
    \caption{
	Isotope correction contribution to the total interaction cross section using own  hit clustering in MWPC2 and MWPC3. Comparison for different quadrant selection in MWPC1. Quadrant four is the preferred one as it is not affected by limited geometric acceptance of TWIN MUSIC.
     }
    \label{fig:iso_corr_quadrants_own_cluster}
\end{figure}

\end{itemize}

\subsection{(Preliminary) Results Total Interaction Cross Section}

\subsection{Quasi-Free Scattering $^{12}$C(p,2p)$^{11}$B}
this is  a reference to \ref{appendix_first}****test
***bottomline: is this really just a subsection?*****
Notes:\newline
\begin{itemize}
\item chapter introduction, first commissioning of p2p reactions at R3B,since first time CALIFA in Final frame and 35\% filled in iPhos
\item what do I have about calibration?
\item channel identification -> Fragment Reconstruction
\item Proton characteristics in Califa
\item inner mom reconstruction, many features here to presen
\item Gamma Spectrum reconstruction / doppler correction (maybe I can use here my own clustering ML stuff?)
\item appendum: separation energy reconstruction? aob? 
\end{itemize}



\subsection{reaction cross section Analysis}

