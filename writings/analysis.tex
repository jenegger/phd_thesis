\section{Analysis - Total Interaction Cross Section $^{12}$C + $^{12}$C - $^{12}$C + $^{12}$C}
This chapter will go through the  analysis step by step from the unpacking stage to the final measurement of the total reaction cross section. It will start by a short overview of the transmission method used for the cross section measurements. The next step is  the selection of clean incoming $^{12}$C isotopes. Following the identification of the carbon isotopes after the target - for the measurement of the charge changing cross section - and as final step the reaction cross section measurement. \newline
All relevant detector related geometrical and efficiency corrections will be adressed and their influence to the final result and its uncertainty will be discussed.
\subsection{Cross Section Measurement via Transmission Method}
In its most generic form cross sections give a measure of the probability that a specific reaction will take place when two or more particles collide. The cross sections  measured in scattering exepriments, as well as the energy and angular distribution of the reaction products, provide information about the dynamics of the interaction between the projectile and the target particle, i.e., about the shape of the interaction potential and the coupling strength.\newline
The cross section $\sigma$ can be derived by looking at the relation between the number of incoming particles and survived particles after collision. For an experiment with fixed target with thickness $z$ and volumetric number density $n$ the number of reacted particles in the infinitesimal thin target layer $dz$ can be expressed as:\newline
%\frac{dI}{dz} = -n\sigmaI 
\begin{equation}
\frac{dI}{dz} = -n \sigma I
%L(t) = \frac{N_f}{\tau_s} exp(-\frac{t}{\tau_f}) + \frac{N_s}{\tau_s} exp(-\frac{t}{\tau_s})
\end{equation}
\subsection{Event Selection}
\subsection{Charge Changing Cross Section Measurement}
\subsection{Geometric Corrections}
\subsection{Isotope Correction - Total Interaction Cross Section}
\subsection{Fine Tuned Geometric Correction}
\subsection{Results}
\subsection{qfs analyis}
\subsection{reaction cross section Analysis}

