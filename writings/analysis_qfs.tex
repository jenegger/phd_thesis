\section{Qualitative Analysis - Quasi-Free Scattering $^{12}$C(p,2p)$^{11}$B}
Until the S444 experiment in 2020 CALIFA consisted out of a prototype frame filled with up to 64 CsI(Tl) crystals. The geometric coverage was therefore poor. For the S444 experiment CALIFA got its final frame and was fully filled in the forward barrel and 35\%filled in the iPhos region, CEPA was not installed yet. With these improvements it was possible to commission qfs-experiments with CALIFA at R3B. In the follow up experiment S467, also in 2020, the first experimental run to study single-particle structures of neutron-rich Ca isotopes via qfs-reactions was carried out.\newline
Even though great improvements in the detector development  were achieved the correction factors to correct for geometric acceptance would be much too high ( $\approx$ 10) for precise cross section measurements for qfs-reactions since the correction factors would in turn rely on a a simplified reaction model. A precise analysis of the acceptance correction factor and the development of a more sophisticated and data driven reaction model is out of the scope of this work. Therefore this analysis focusses on the methods of qfs-reaction identification and the extraction of the key informations dicussed in section \ref{sec:qfs_theo}. TODO: add more stuff here...\newline
\subsection{Setup-Calibration}
For setup description refer to section \ref{sec:analysis_cross_sec}. For all detectors except the SOFIA (Study On FIssion with Aladin) Time of Flight Wall the calibration parameters investigated by the respective detector-expert group were adopted. Herefore we will subsequently only describe the Time of Flight Wall calibration in this subsection.
\subsubsection{Flight-Path Reconstruction - SOFIA Time of Flight Wall}
The procedure to calibrate the Time of Flight (ToF) Wall involves beforehand a precise flight-path reconstruction of the projectile from the entrance of Cave C downstream to the ToF Wall. 
-> sweep runs
-> explain how calibration was done
\subsection{Event Selection}
no major efforts on th event selection done, look up the code.
\subsection{Fragment Identification}
->order of reconstruction: first fragment, critical! then the two protons

\subsection{QFS-Protons}

\subsection{Reconstruction of excited$ ^{11}$B states}

\subsection{Separation energy, correlation fragment protons, aob..}
