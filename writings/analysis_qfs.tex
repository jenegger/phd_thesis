\section{Qualitative Analysis - Quasi-Free Scattering $^{12}$C(p,2p)$^{11}$B}
Until the S444 experiment in 2020 CALIFA consisted out of a prototype frame filled with up to 64 CsI(Tl) crystals. The geometric coverage was therefore poor. For the S444 experiment CALIFA got its final frame and was fully filled in the forward barrel and 35\%filled in the iPhos region, CEPA was not installed yet. With these improvements it was possible to commission qfs-experiments with CALIFA at R3B. In the follow up experiment S467, also in 2020, the first experimental run to study single-particle structures of neutron-rich Ca isotopes via qfs-reactions was carried out.\newline
Even though great improvements in the detector development  were achieved the correction factors to correct for geometric acceptance would be much too high ( $\approx$ 10) for precise cross section measurements for qfs-reactions since the correction factors would in turn rely on a a simplified reaction model. A precise analysis of the acceptance correction factor and the development of a more sophisticated and data driven reaction model is out of the scope of this work. Therefore this analysis focusses on the methods of qfs-reaction identification and the extraction of the key informations dicussed in section \ref{sec:qfs_theo}. TODO: add more stuff here...\newline
\subsection{Setup-Calibration}
For setup description refer to section \ref{sec:analysis_cross_sec}. For all detectors except the SOFIA (Study On FIssion with Aladin) Time of Flight Wall the calibration parameters investigated by the respective detector-expert group were adopted. Herefore we will subsequently only describe the Time of Flight Wall calibration in this subsection.
\subsubsection{Flight-Path Reconstruction - SOFIA Time of Flight Wall}
The procedure to calibrate the Time of Flight (ToF) Wall involves beforehand a precise flight-path reconstruction of the projectile from the entrance of Cave C downstream to the ToF Wall. Since only one tracking detector downstream to GLAD was in operation for the S444 experiment no angle of the deflected fragment/beam $\theta_{out}$ could be directly extracted. Herefore an advanced tracking algorithm was developed, motivated from ref. \cite{bertini2013study} (section 3.4). 
\begin{itemize}
\item The first step is to measure the scattering angle after target, $\theta_{in}$, and draw an extended line from the target position through the effective magnetic field of GLAD.  
\item Draw a trajectory line from the hit position in MWPC3 to the intersection point C (see figure \ref{fig:sub1_reco_path}) of the "kick-plane-line" and the reconstructed and extended track line upstream to GLAD.
\item Now sweep along the reconstructed track line upstream to GLAD. For each step a value for $\theta_{out}$ and $d1-d2$ is gathered, see figure \ref{fig:sub2_reco_path}.
\item As in figure \ref{fig:sub2_reco_path} shown, fit the data-points from previous step with a linear fit function and find the corresponding $\theta_{out}$ value where the fit line intersects with the abscissa. This corresponds to the case where $d1 = d2$. This is the approximated "kickpoint" in GLAD. 
\item Previous steps need to be executed for all events accordingly.
\end{itemize}
\begin{figure}[ht]
    \centering
    % First subfigure
    \begin{subfigure}[b]{0.70\textwidth}
        \includegraphics[width=\textwidth]{Figures/kick_point_algorithm.png}
        \caption{Flightpath reconstruction with "Leff" being the effective active width of the magnetic field of GLAD}
        \label{fig:sub1_reco_path}
    \end{subfigure}
    %\hfill % Optional: adds horizontal space between the figures
    % Second subfigure
    \begin{subfigure}[b]{0.25\textwidth}
        \includegraphics[width=\textwidth]{Figures/intersection_algorithm.png}
        \caption{$\theta_{out}$ approximation. For detailed information see text.}
        \label{fig:sub2_reco_path}
    \end{subfigure}

    \caption{Flightpath tracking and reconstruction of the fragment/beam after the target.}
    \label{fig:reco_path}
\end{figure}

Now that the scattering angle after the target $\theta_{in}$ and the angle after GLAD, $\theta_{out}$ is known and the position outside the magnetic field of GLAD are fixed, with $(z_1,x_1)$  the entrance point on the GLAD field and $(z_2,x_2)$ the exit point, the bending radius $r$ from the magnetic deflection can be determined:
\begin{equation}\label{eq:rho_glad}
r = \frac{L_{eff}}{2\,\cdot sin\left(\frac{\theta_{in}+\theta_{out}}{2}\right)\,\cdot cos\delta}
\end{equation}
where $L_{eff}$ is the effective active with of the magnetic field of GLAD, which correspons $\approx 2.06\,m$. This value could also be verified by extracting $L_{eff}$ from the formula for the magnetic rigidity $ B\cdot r = \gamma\beta \, m /q$ for empty target runs with known values of the B-field\footnote{For detector calibration, primarly for the ToFWall, we had several "empty sweep runs", with empty target and different B-field strenght settings. Those runs were optimal to validate $L_{eff}$.}.\newline
The angle $\delta$ in equation \ref{eq:rho_glad} is given by the trajectory line going through $(z_1,x_1)$/$(z_2,x_2)$ and the line parallel to the GLAD magnet width $L_{eff}$.\newline
A detailed derivation of equation \ref{eq:rho_glad} can be found in appendix \ref{app:flightpath}.
The arc trajectory $l_{GLAD}$ of the fragment within GLAD can be reconstructed using the bending radius $r$ and the entry and exit points $(z_1,x_1)$/$(z_2,x_2)$:
\begin{equation}\label{eq:arc}
l_{GLAD} = r \cdot \omega\qquad \text{with}\qquad \omega = 2\cdot asin(t_{1/2}{(2\cdot r)})
\end{equation}
where $\omega$ is the central angle and  $t_{1/2}$ is the cord length between $(z_1,x_1)$ and $(z_2,x_2)$.\newline
At this stage the full pathlength $L$ form the Start detector to the ToFW is fixed:
\begin{description}
\item \textbf{From Start to the target $l_{ST}$:} For this path section a straight flightpath parallel to the z-coordinate is assumed. The pathlength is taken from the position measurements of the Start detector and the target accordingly ($=118.3\,cm$). 
\item \textbf{Target to GLAD entrance point $(z_1,x_1)$, $l_{in}$:}For the  exact position assignment of $(z_1,x_1)$ both MWPC1 and MWPC2 position measurements have been calibrated with empty target runs by including an offset valuein order to center the x-position in both detectors around zero. From the position measurement of the central position of GLAD, its tilting angle $\alpha$ and $L_{eff}$ the intersection point of the fragment trajectory before GLAD and the "effective GLAD magnetic field rectangle", $(z_1,x_1)$, can be determined and with it the according flightpath section $l_{in}$.
\item \textbf{Arc trajectory within GLAD, ${l_{GLAD}}$:}This flightpath passage is determined by reconstructing $(z_1,x_1)$ and $(z_2,x_2)$, as it has been described in detail in the previous section. Hence the magnetic bending radius can be determined, see equation \ref{eq:rho_glad}, and finally the arc trajectory ${l_{GLAD}}$ as in equation \ref{eq:arc}.
\item \textbf{GLAD exit point $(z_2,x_2)$ to ToFW, $l_{out}$:}The hit position $(z_{MWPC3},x_{MWPC3})$ in MWPC3 is given by the reconstructed hit position in this detector and the position measurement of the detector itself. The exit point $(z_2,x_2)$ was reconstructed in the previous steps. Hence the trajectory line from $(z_2,x_2)$ to $(z_{MWPC3},x_{MWPC3})$ is fixed and is expanded to the intersection with the ToFW plane for the concluding $l_{out}$ measurement.  
\end{description}
The resulting flightpath $L$ recombines from:
\begin{align*}
L &= l_{ST}+l_{in}+l_{GLAD} + l_{out} 
\end{align*}
In the flightpath reconstruction the deflections in the y-dimension were omitted as the angular straggling in the target is small (TODO: give sigma value) and since the deflection of the fragment within GLAD is independent from its  y-position this contribution can be disregarded.


\subsubsection{Time of Flight Calibration -  SOFIA Start \& Time of Flight Wall}
For the time of flight measurement in the S444 setup the time is recorded and digitized by the VFTX,  VME-FPGA Time-to-Digital Converter (TDC) Modules based on tapped delay line (TDL) TDCs\cite{bayer2009development}. These modules provide for each detected signal a coarse time, is determined by counting cycles of a 200 MHz readout clock, resulting in a 5 ns binning resolution, and a fine time, which is obtained using an FPGA-based Time-to-Digital Converter (TDC), which employs a tapped delay line (TDL). In this approach, the signal propagates through a series of delayed logic modules within the FPGA until the subsequent clock cycle terminates the sampling process. The number of delay elements traversed by the signal is used to compute the time difference between the signal onset and the end of the clock cycle. The translation of the resulting fine time, with reasonable assumption of an uniformly distributed fine time, is achieved via a calibrated linear function. This procedure assigns to each preamplifier signal in the start detector (left/right) and the ToF Wall (up/down) a calibrated raw time \textit{RawTimeNs}:
\begin{align*}
RawTimeNs &= coarse\_time\_clocks  \cdot 5ns + offset[fine\_time]
\end{align*}
Finally, the raw time of flight is reconstructed by combinding all four time measurements:
\begin{align*}
RawTof &= 0.5*(RawTimeNs\_{i,down}+RawTimeNs\_{i,up}) - 0.5*(RawTimeNs\_{start,right}+RawTimeNs{start,left})\newline
\end{align*}
where \textit{i} $(\in 0...27)$ refers to the scintillator number of the ToF Wall. As 

%TODO: shortly explain VFTX, coarse time and fine time. \newline
%When going from mapped to cal you do RawTimeNs =  (coarse time)*5ns + offset[fine_time], this is no physical time\newline
%as next step you get the raw time:\newline
%RawTof = 0.5*(RawTimeNs_{i,down}+RawTimeNs_{i,up}) - 0.5*(RawTimeNs_{start,right}+RawTimeNs{start,left})\newline
%now real calibration starts:\newline
%To calculate the offset for each pad in ToFWall you have to do:\newline
%delta_offset = mean_pl[i]/v_beam  - raw_time_of_flight_ns_mean[i], where v_beam = 214,2mm/ns (you get this also from atima), using empty target runs
%To get final time resolution you do:
%mean_pl[i]/(pathlengt_eventvise[i]/calib_time_eventvise  now calib time has the delta_offset included
%time res, was somewhere around 90ps.




\subsection{Event Selection}
no major efforts on th event selection done, look up the code.
\subsection{Fragment Identification}
->order of reconstruction: first fragment, critical! then the two protons

\subsection{QFS-Protons}

\subsection{Reconstruction of excited$ ^{11}$B states}

\subsection{Separation energy, correlation fragment protons, aob..}
