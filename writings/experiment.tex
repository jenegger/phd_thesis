\section{Experiment}
The present commissioning experiment was performed in 2020 at the FAIR Facility at GSI (Gesellschaft f\"ur Schwerionenforschung) in Darmstadt (Germany). The GSI Helmholtzzentrum für Schwerionenforschung operates a unique accelerator facility for heavy ions and focuses on several cutting-edge research fields. These include:\newline
\begin{enumerate}
\item \textbf{Nuclear Physics}: Studying the properties of atomic nuclei, exploring the forces that bind protons and neutrons, and investigating exotic nuclei far from stability.
\item \textbf{Hadron and Quark Matter}: Investigating the behavior of hadrons (particles made of quarks) and the state of matter under extreme conditions, such as those found in neutron stars or during the early universe.
\item \textbf{Atomic Physics}: Examining the structure and dynamics of atoms, including highly charged ions, to understand fundamental atomic interactions and refine quantum electrodynamics.
\item \textbf{Plasma Physics}: Creating and analyzing high-energy-density plasmas to simulate conditions found in stellar interiors and other astrophysical phenomena.
\item \textbf{Biophysics and Medical Research}: Exploring the effects of ion beams on biological systems for applications in cancer therapy, particularly using heavy ion therapy, and studying radiation protection for space missions.
\item \textbf{Materials Research}: Investigating the response of materials to high radiation doses to develop more resilient materials for use in various technologies, including nuclear reactors and space exploration.
\end{enumerate}
\subsection{GSI facility}
\subsection{R3B Setup}
\subsubsection{Detector Setup in S444 Commmissioning Experiment 2020}


