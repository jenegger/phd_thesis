\section{Experiment}
The present commissioning experiment was performed in 2020 at the FAIR Facility at GSI (Gesellschaft f\"ur Schwerionenforschung) in Darmstadt (Germany). The GSI Helmholtzzentrum für Schwerionenforschung operates a unique accelerator facility for heavy ions and focuses on several cutting-edge research fields. These include:\newline
\begin{enumerate}
\item \textbf{Nuclear Physics}: Studying the properties of atomic nuclei, exploring the forces that bind protons and neutrons, and investigating exotic nuclei far from stability.
\item \textbf{Hadron and Quark Matter}: Investigating the behavior of hadrons (particles made of quarks) and the state of matter under extreme conditions, such as those found in neutron stars or during the early universe.
\item \textbf{Atomic Physics}: Examining the structure and dynamics of atoms, including highly charged ions, to understand fundamental atomic interactions and refine quantum electrodynamics.
\item \textbf{Plasma Physics}: Creating and analyzing high-energy-density plasmas to simulate conditions found in stellar interiors and other astrophysical phenomena.
\item \textbf{Biophysics and Medical Research}: Exploring the effects of ion beams on biological systems for applications in cancer therapy, particularly using heavy ion therapy, and studying radiation protection for space missions.
\item \textbf{Materials Research}: Investigating the response of materials to high radiation doses to develop more resilient materials for use in various technologies, including nuclear reactors and space exploration.
\end{enumerate}
\subsection{GSI facility}
The GSI Helmholzzentrum f\"ur Schwerionenforschung located at Darmstadt has a long history of research.... tell something about the beginnnings, first really heavy elements found there.\newline
%Tell about the central apparatus: Linear Accelerator UNILAC, Ring Accelerator SIS 18, FRS,  and the different experimental halls, see: https://web.archive.org/web/20141222164000/https://www.gsi.de/en/research/accelerator_facility.htm?nr=%2Fproc%2Fself%2Fe
The GSI Helmholtzzenrum f\"ur Schwerionenforschung GmbH was founded in 1969 (as "Gesellschaft f\"ur Schwerionenforschung mbH) looks back on a successful research history. In the time between 1981 and 2010 six new  superheavy elements were discovered. \newline
In the medical research field GSI has developed advanced cancer therapy techniques using heavy ion beams which target tumors with high precision, minimizing damage to surrounding healthy tissues.\newline
Along with those groundbreaking discoveries in research the facility at GSI has always been an inspiring source of drive for new technologies.\newline
The key devices/apparatus which enable to carry out experiments with heavy ions at GSI are:
%cite from:
%https://www.gsi.de/en/researchaccelerators/accelerator_facility
important to mention: GSI is the only facility with heavy ions in the world
The starting point for the production of relativistic heavy ions at GSI is the ion source where ions are generated by stripping electrons off the shell of the atoms. Depending on the experimental needs the ion sources at GSI are able to produce ions of many different kinds of elements (up to Uranium).\newline
%cite here: https://www.gsi.de/en/researchaccelerators/accelerator_facility/ion_sources
On the first acceleration stage the stable primary ions are injected from the ion source into the UNIversal Linear Accelerator (UNILAC). On a length of 120 meters ions are accelerated up to maximum energy of 11.4 AMeV. The low energy beam is now injected into the ring accelerator SIS18 (Schwerionensynchotron 18). Here the ion beam is further accelerated up to 4.7 GeV/u (for protons) / 1 GeV/u (for Uranium). The magnets and  the ultra-high vacuum (~ 10⁻9 Pa) keep the ions well on their circular path (SIS18 has a circumference of 216 meters). For the production of rare heavy isotopes the primary ion beam from SIS18 can be impinged on a light nuclear target, e.g beryllium, the so called production target. These secondary beams of radioactive isotopes can be either stored in the experimental storage ring (ESR) for later use or transferred to the FRagment Separator (FRS). The FRS as a high-resolution magnetic spectrometer is capable to precisely select specific isotopes and to forward the desired beam of exotic relativistic nuclei to the various experiments or direct it to the ESR for later use.\newline
%cite here: https://www.gsi.de/en/researchaccelerators/accelerator_facility/ring_accelerator
\subsubsection{FAIR Project}
The FAIR (Facility for Antiproton and Ion Research) situated next to the GSI will be one of the most complex and largest accelerator facilities in the world. The construction of the superconducting ring accelerator SIS100 with a circumference of 1.1 km, storage rings and experiment sites begun in the summer of 2017. Commissioning is planned in 2025 (?). Early Science. 
\subsection{R3B Setup}
\subsubsection{Detector Setup in S444 Commmissioning Experiment 2020}
\subsubsection{Calibration of the Detector Systems}



