\section{Experiment}
The present commissioning experiment was performed in 2020 at the FAIR Facility at GSI (Gesellschaft f\"ur Schwerionenforschung) in Darmstadt (Germany). The GSI Helmholtzzentrum für Schwerionenforschung operates a unique accelerator facility for heavy ions and focuses on several cutting-edge research fields. These include:\newline
\begin{enumerate}
\item \textbf{Nuclear Physics}: Studying the properties of atomic nuclei, exploring the forces that bind protons and neutrons, and investigating exotic nuclei far from stability.
\item \textbf{Hadron and Quark Matter}: Investigating the behavior of hadrons (particles made of quarks) and the state of matter under extreme conditions, such as those found in neutron stars or during the early universe.
\item \textbf{Atomic Physics}: Examining the structure and dynamics of atoms, including highly charged ions, to understand fundamental atomic interactions and refine quantum electrodynamics.
\item \textbf{Plasma Physics}: Creating and analyzing high-energy-density plasmas to simulate conditions found in stellar interiors and other astrophysical phenomena.
\item \textbf{Biophysics and Medical Research}: Exploring the effects of ion beams on biological systems for applications in cancer therapy, particularly using heavy ion therapy, and studying radiation protection for space missions.
\item \textbf{Materials Research}: Investigating the response of materials to high radiation doses to develop more resilient materials for use in various technologies, including nuclear reactors and space exploration.
\end{enumerate}
\subsection{GSI facility}
The GSI Helmholzzentrum f\"ur Schwerionenforschung located at Darmstadt has a long history of research.... tell something about the beginnnings, first really heavy elements found there.\newline
%Tell about the central apparatus: Linear Accelerator UNILAC, Ring Accelerator SIS 18, FRS,  and the different experimental halls, see: https://web.archive.org/web/20141222164000/https://www.gsi.de/en/research/accelerator_facility.htm?nr=%2Fproc%2Fself%2Fe
The GSI Helmholtzzenrum f\"ur Schwerionenforschung GmbH was founded in 1969 (as "Gesellschaft f\"ur Schwerionenforschung mbH) looks back on a successful research history. In the time between 1981 and 2010 six new  superheavy elements were discovered. \newline
In the medical research field GSI has developed advanced cancer therapy techniques using heavy ion beams which target tumors with high precision, minimizing damage to surrounding healthy tissues.\newline
Along with those groundbreaking discoveries in research the facility at GSI has always been an inspiring source of drive for new technologies.\newline
The key devices/apparatus which enable to carry out experiments with heavy ions at GSI are:
%cite from:
%https://www.gsi.de/en/researchaccelerators/accelerator_facility
important to mention: GSI is the only facility with heavy ions in the world
The starting point for the production of relativistic heavy ions at GSI is the ion source where ions are generated by stripping electrons off the shell of the atoms. Depending on the experimental needs the ion sources at GSI are able to produce ions of many different kinds of elements (up to Uranium).\newline
%cite here: https://www.gsi.de/en/researchaccelerators/accelerator_facility/ion_sources
On the first acceleration stage the stable primary ions are injected from the ion source into the UNIversal Linear Accelerator (UNILAC). On a length of 120 meters ions are accelerated up to maximum energy of 11.4 AMeV. The low energy beam is now injected into the ring accelerator SIS18 (Schwerionensynchotron 18). Here the ion beam is further accelerated up to 4.7 GeV/u (for protons) / 1 GeV/u (for Uranium). The magnets and  the ultra-high vacuum (~ 10⁻9 Pa) keep the ions well on their circular path (SIS18 has a circumference of 216 meters). For the production of rare heavy isotopes the primary ion beam from SIS18 can be impinged on a light nuclear target, e.g beryllium, the so called production target. These secondary beams of radioactive isotopes can be either stored in the experimental storage ring (ESR) for later use or transferred to the FRagment Separator (FRS). The FRS as a high-resolution magnetic spectrometer is capable to precisely select specific isotopes and to forward the desired beam of exotic relativistic nuclei to the various experiments or direct it to the ESR for later use.\newline
%cite here: https://www.gsi.de/en/researchaccelerators/accelerator_facility/ring_accelerator
\subsubsection{FAIR Project}
The FAIR (Facility for Antiproton and Ion Research) situated next to the GSI will be one of the most complex and largest accelerator facilities in the world. The construction of the superconducting ring accelerator SIS100 with a circumference of 1.1 km, storage rings and experiment sites begun in the summer of 2017. Commissioning is planned in 2025 (?). Early Science. Before the commmissioning of the ring accelerator SIS100 several prioritized experiments with large impact in the scientific world will take place in the newly built experimental halls, such as experiments with the R3BSetup in the High Energy Cave (HEC).\newline
 
\subsection{R3B Setup}
The R3B (Reactions with Relativistic Radioactive Beams) experiment in Cave C at the GSI Helmholtz Centre for Heavy Ion Research in Germany is a cutting-edge research experiment focused on the study of nuclear reactions and structure using high-energy radioactive ion beams. The experiment aims to investigate exotic nuclei far from stability, offering insights into the fundamental properties of nuclear matter, nucleosynthesis processes, and the forces governing nuclear interactions. A schematic overview of the R3B Setup can be seen in Figure blabla. \newline
The short living (neutron rich) isotopes are injected to the Cave C from the FRS, which preselects as mass spectrometer the isotopes of interest, and impinge on a fixed target. The R3B setup is designed for kinematically complete reaction studies. To fulfill this requirement the incoming ions are tracked and identified on an event-by-event basis by dedicated detectors in the FRS via time-of-flight and deltaE measurement techniques. Depending on the settings and composition of the incoming ion beam different type of reactions take place in the target area with a large variety of reaction products: heavy ions (as producs from fission/spallation reactions), neutrons, light charged particles and gamma rays. For the detection of gammas and light charged ions from reactions with the target the dedicated CALIFA calorimeter (see more in section blabla) and various tracking detectors are installed in the target region. The GLAD (GSI Large Acceptance Dipole) magnet, located at the center of the Cave C, acts as mass spectrometer for the forward boosted charged reaction residues. The magnetic rigidity of the charged reaction residues is measured by a combination tracking detectors and a time of flight wall after the GLAD magnet. This allows to identify the charged reaction residues and their momenta. For the detection of the neutrons, not deflected by the magnetic field of the GLAD magnet, the new array neutron detector (NeuLAND) is positioned after GLAD on the zero degree line with the incoming ion beam.\newline
The high flexibility of the R3B Setup, it can be operated with 
The combination of the large spectrum of incoming ion beams in a broad energy range provided by the FRS facility and the high flexibility of the R3B Setup with state of the art detectors for the specific physics-studies of interest makes it to an attractive play-ground for experimental astro-physics.\newline
\subsection{Detector Setup in S444 Commmissioning Experiment 2020}
The S444 Experiment (successor experiment of the FAIR Phase-0 program in 2019, ref to Lukas Ponnath Thesis) for the commissioning of the CALIFA Calorimeter in its final mechanical design took place in February 2020. The choice to operate with stable 12C primary beam with four beam energy settings - 400/550/650/800 AMeV  gave the opportunity to use it as preparation for the following up S467 experimental run with neutron-rich Ca isotopes as medium-heavy incoming beam. The detectors for positional tracking, charge identification and time measurement were provided by the SOFIA(Study on Fission with Aladin, make footnote that ALADIN was the precessor or GLAD) collaboration. These detectors are optimized for fission experiments with medium to heavy reaction fragments. As for the S444 experiment with primary 12C incoming beam no fission reaction with multiple heavy charged fragments is expected the Sofia detectors were adapted accordingly (e.g. only one of the four sections of the Twin-Music Ionisation chamber was operated, see more in chapter Twin).\newline
Tell that the detectors were mostly in air...
\subsubsection{Sofia Start Detector}
\subsubsection{Multi Wire Proportional Chambers (MWPC)}
\subsubsection{Ionisation Chambers - R3BMusic/TWIM Music}
\subsubsection{GLAD Magnet}
\subsubsection{CALIFA Calorimeter}
\subsubsection{Sofia Time of Flight Wall}
\subsubsection{NeuLAND Detector}

\subsubsection{Calibration of the Detector Systems}



