\section{Introduction}
Talk something about first ideas from Greeks of the structure of matter /historical part
Comparisons between atomic model and nuclear models
And give a short overview about what you will talk in this introduction part and how related to your thesis
\begin{itemize}
\item 
\end{itemize}
\subsection{Nuclear Structure Models}
Influenced by the good results/predictions of the Fermi gas model -> same model applied to nucleons in nucleus
-> talk about the key features and limitations of this model
->liquid drop model -> can well postulate the binding energy (model from experimental observations)
From the success of the atmic shell model -> nuclear shell model
-> describes well shell closures->magic numbers
->picture and description of shell model
->limitations
-> maybe mention optical model -> as it is used to predict cross section in elastic scattering reactions
->mention that  there are also other models as mean field, cluster models, ab initio models, collective models which exceed the topics of this thesis, but give some references to it.
\subsection{R$^3$B Experiment - "The Universe in the Lab"}
Central formula is the equation of states (EOS). We want to study it within a wide range. 
\begin{itemize}
\item in astrophysics NS are herefore of special interest. Measuring their mass and radius -> this relation depends on the EOS. 
\item since NS are highly asymmetric matter (neutrons only) they are a good playground to test the different models. Since they are all consistent at rho0 (real world) they diverge for higher matter asymmetry. 
\item plot of some models
\item formula of tailor expansion of Equation of state
\item at FAIR we want to study  nuclear matter under extreme conditions,highly asymmetric matter. 
\item constraining the symmetry energy
\item this can be done via neutron removal cross section, cite T. Aumann, Bertulani etc.
\item show picture of sensitiveness of this method, shortly mention that L was previously constrained in PREX experiments(give some literature for further readings)
\item the predicted models rely on the input from reaction mechanisms -> 12C+12C good to probe the underlying reaction mechanisms-> one part of the thesis
\end{itemize}
Moreover to study the single particle states of nucleons inside nucleus, exspecially when going toward exotic nuclei-> quasi free scattering experiments.-> refer to the theory of those scattering experiments
This was the first time we had our CALIFA CALORIMETER in its final frame. 
Refer to the application section. \newline
To point out is the pilot experiment we had in 2021 to study fission via qfs. This could have a large impact in the understanding of the dynamics of fission far off the line of stability. \newline
Explain the fission via qfs a little bit more in detail here.\newline
This could explain maybe better the element abundances we have and if there is any island of stability at Z > 126.

