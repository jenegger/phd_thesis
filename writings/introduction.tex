\section{Introduction}
The elemental abundances observed throughout the universe are a fingerprint of the formation and evolution of astrophysical objects. Understanding the origin and distribution of these elements requires a robust theoretical framework based on nuclear structure models. The synthesis of elements -- from the lightest nuclei formed during Big Bang nucleosynthesis to the heaviest transuranic elements produced in explosive astrophysical environments -- depends critically on the properties of atomic nuclei.\newline
The foundation for modern nuclear structure theory was laid by the atomic shell model, first introduced by Niels Bohr in 1913. This model, while initially developed for electrons in atoms, inspired the formulation of the nuclear shell model by Maria Goeppert Mayer and J. Hans D. Jensen in 1949. Their key innovation was the inclusion of a significant spin-orbit interaction term in the nuclear mean-field potential. Unlike the atomic case -- where spin-orbit coupling appears as a relatively small fine-structure correction -- in the nuclear shell model, the spin-orbit interaction is of comparable magnitude to the primary energy level gaps and carries an inverted sign. This results in the $j = l + 1/2$ states lying energetically below the $j = l - 1/2$  states.\newline
This critical adjustment enabled the nuclear shell model to accurately reproduce the so-called "magic numbers" (2, 8, 20, 28, 50, 82, and 126), which correspond to closed shells of protons or neutrons associated with exceptional nuclear stability. Notable examples include isotopes such as those of thin (Z = 50) and  lead ( Z = 82), where large energy gaps between filled and unfilled shells lead to enhanced binding energies and structural rigidity.\newline
While the shell model has proven highly successful in describing nuclei close to stability, recent experiments have shown, that the magic shells can be substantially modified in case of exotic species, like $^{32}$Mg \cite{wimmer2010discovery} or $^{28}$O \cite{kanungo2009one}. In case of more heavy regions, experimental data remain scarce, and the shell model must be extended or augmented to account for collective phenomena and configuration mixing. These nuclei play a central role in rapid neutron-capture processes (r-process), which are responsible for the formation of approximately half of the heavy elements beyond iron. In particular, the fate of the r-process path -- its termination via fission and the possible emergence of a predicted "island of stability" in the superheavy region -- depends critically on the structural properties of such exotic nuclei.\newline
The first part of this chapter provides a concise overview of the development of nuclear structure models. Emphasis is placed on the evolution of model concepts in response to the experimental and astrophysical challenges of their time. The second part introduces the scientific objectives and research program of the R$^3$B experiment. In particular, we highlight the role of nuclear structure studies in constraining the nuclear equation of state (EOS), especially for highly asymmetric nuclear matter. Nuclear scattering reactions at R$^3$B offer an effective probe of these systems, and their theoretical treatment -- along with a general overview of reaction mechanisms -- will be presented in chapter \ref{sec:reac_model}.

\subsection{Nuclear Structure Models}
Influenced by the good results/predictions of the Fermi gas model -> same model applied to nucleons in nucleus
-> talk about the key features and limitations of this model
->liquid drop model -> can well postulate the binding energy (model from experimental observations)
From the success of the atmic shell model -> nuclear shell model
-> describes well shell closures->magic numbers
->picture and description of shell model
->limitations
-> maybe mention optical model -> as it is used to predict cross section in elastic scattering reactions
->mention that  there are also other models as mean field, cluster models, ab initio models, collective models which exceed the topics of this thesis, but give some references to it.
\subsection{R$^3$B Experiment - "The Universe in the Lab"}
Central formula is the equation of states (EOS). We want to study it within a wide range. 
\begin{itemize}
\item in astrophysics NS are herefore of special interest. Measuring their mass and radius -> this relation depends on the EOS. 
\item since NS are highly asymmetric matter (neutrons only) they are a good playground to test the different models. Since they are all consistent at rho0 (real world) they diverge for higher matter asymmetry. 
\item plot of some models
\item formula of tailor expansion of Equation of state
\item at FAIR we want to study  nuclear matter under extreme conditions,highly asymmetric matter. 
\item constraining the symmetry energy
\item this can be done via neutron removal cross section, cite T. Aumann, Bertulani etc.
\item show picture of sensitiveness of this method, shortly mention that L was previously constrained in PREX experiments(give some literature for further readings)
\item the predicted models rely on the input from reaction mechanisms -> 12C+12C good to probe the underlying reaction mechanisms-> one part of the thesis
\end{itemize}
Moreover to study the single particle states of nucleons inside nucleus, exspecially when going toward exotic nuclei-> quasi free scattering experiments.-> refer to the theory of those scattering experiments
This was the first time we had our CALIFA CALORIMETER in its final frame. 
Refer to the application section. \newline
To point out is the pilot experiment we had in 2021 to study fission via qfs. This could have a large impact in the understanding of the dynamics of fission far off the line of stability. \newline
Explain the fission via qfs a little bit more in detail here.\newline
This could explain maybe better the element abundances we have and if there is any island of stability at Z > 126.

