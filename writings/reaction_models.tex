\section{Reaction Models}
\subsection{Glauber Theory}
\subsection{Quasi-Free Scattering (QFS) Reactions}
Quasi-free scattering (QFS) reactions, as a subset of direct reactions, are processes where a projectile nucleon interacts with a target nucleon in a single, fast, and highly localized event. The timescale of these interactions is extremely short, approximately $10^{-21}$ s, implying that the relative kinetic energies of the participants are high, typically $\gtrsim$ 100 AMeV.\newline
In QFS experiments conducted in direct kinematics, a proton serves as the projectile, colliding with a nucleon or a cluster within the target nucleus. Conversely, in inverse kinematics, the nucleus of interest becomes the projectile, while a proton or proton-like particle is used as the target. Despite these differing setups, both approaches are fundamentally equivalent, differing only due to a Lorentz transformation of the reference frame.\newline
As the name implies, QFS reactions are conceptually similar to free nucleon-nucleon scattering, with the primary approximation being that the influence of the residual nucleus is neglected to first order. This approximation simplifies the theoretical description of the process, allowing to treat the interaction as a two-body problem within a (only weakly interfering) nuclear environment.\newline
The first experiments confirming the existence of quasi-free scattering processes were conducted at the University of California, Berkeley, in 1952 by O. Chamberlain and Emilio Segr\`e\cite{chamberlain1952proton}. In their study, lithium nuclei were bombarded with 350 MeV protons, and coincident proton pairs were observed with an opening angle of approximately 90$^{\circ}$. That same year, J.B. Cladis, W.N. Hess, and B.J. Moyer published results on the scattering of 340 MeV protons on deuterium and carbon targets\cite{cladis1952nucleon}, further substantiating the phenomenon.\newline
In 1957, Tyren, Maris, and Hillman designed an experiment aimed at fully characterizing proton-proton collisions within the quasi-free scattering framework\cite{MARIS19581}. Their results validated the assumption of a direct and clean interaction between the projectile and the target nucleon, free from significant distortions caused by the surrounding nucleus. Furthermore, these experiments demonstrated that QFS reactions could serve as a powerful tool for probing nuclear structure and testing predictions of the shell model. Specifically, they enabled the study of key nuclear parameters such as spin-orbit splitting and energy differences between nuclear shells and here fore to probe the shell evolution. \newline
In QFS reactions, these parameters can be extracted by analyzing the reaction products, which include the two correlated outgoing protons, the residual nucleus, and any gamma rays emitted during de-excitation of the residual nucleus. The detailed measurements of these observables provide critical insights into the underlying nuclear structure and dynamics.\newline
The experimental discoveries and theoretical insights from these early studies catalyzed significant advancements in the theoretical modeling of QFS reactions. These models have since become essential tools for understanding nucleon-nucleon interactions within the nuclear medium and for refining our knowledge of nuclear structure and reaction mechanisms.\newline
